\documentclass[a4paper, 12pt, twoside]{article}
\usepackage[a4paper, top=2.5cm, bottom=2.5cm, left=2.5cm, right=2.5 cm, bindingoffset=0.5cm]{geometry}
\usepackage[onehalfspacing]{setspace}
\usepackage[utf8]{inputenc}
\usepackage[polish]{babel}
\usepackage[T1]{fontenc}
\usepackage{graphicx}
\usepackage{ragged2e}
\usepackage{indentfirst}
\usepackage{csquotes}
\usepackage[backend=biber, maxbibnames=30]{biblatex}
\usepackage{amsmath}
\usepackage{todonotes}

\addbibresource{inz.bib}
\graphicspath{ {./images/} }

\title{System IIoT do zbierania danych pomiarowych}
\date{}

\begin{document}

\clearpage\maketitle %no page number on title page
\thispagestyle{empty}

\listoftodos

\newpage

\section{Wstęp}

\section{Analiza tematu}

\subsection{Systemy informatyczne do zbierania danych pomiarowych} \label{system-do-zbierania-danych}

\subsubsection*{Systemy informatyczne}

Na potrzeby omówienia zagadnienia \emph{systemu do zbierania danych pomiarowych} oraz
rozpatrzenia jego możliwych zastosowań konieczne jest wprowadzenie pewnych 
pojęć podstawowych.
Jednym z nich jest \textbf{system}. Zgodnie z encyklopedią systemem
nazywamy wyizolowany z otoczenia zespół elementów przeznaczonych
do realizacji określonego zadania \cite{system}. Szczególnym przykładem systemu jest
\textbf{system komputerowy}. Jego elementy stanowią komputery, 
urządzenia peryferyjne
i oprogramowanie podstawowe (np. system operacyjny) \cite{system-komputerowy}.
Systemy komputerowe, sieci oraz oprogramowanie stanowią razem 
\textbf{systemy informatyczne}, których celem zastosowania jest 
przetwarzanie informacji \cite{system-informatyczny}. 
\textbf{Przetwarzaniem informacji} nazywa się proces akwizycji, utrwalania, udostępniania,
organizacji, interpretacji i wizualizacji informacji \cite{information-science}, 
\cite{information-processing}. Warto w~tym miejscu również doprecyzować pojęcia
\textbf{informacji} oraz \textbf{danych}. Dane są pewną reprezentacją
faktów skodyfikowaną w odpowiedni sposób. Same dane nie posiadają jednak celu.
Znaczenia nabierają te dane, które zostaną zinterpretowane przez odbiorcę 
i~zmniejszają jego niewiedzę.
Przyswojone w ten sposób dane stają się informacją \cite{dane-informacja-wiedza}.

\subsubsection*{Systemy do zbierania danych pomiarowych}

Jednym z możliwych źródeł informacji w systemie informatycznym mogą być 
\textbf{systemy do zbierania danych pomiarowych} (system akwizycji danych, ang. \emph{Data Acquisition Systems}, w skrócie: DAQ).
Są to systemy, których rolą jest gromadzenie w formie cyfrowej informacji opisujących 
zjawiska fizyczne. Systemy DAQ tworzą między innymi następujące komponenty
o odpowiednich przeznaczeniach:
\begin{itemize}
    \item czujniki (sensory, ang. \emph{sensors}): przetwarzanie zjawisk fizycznych na analogowy sygnał elektryczny,
    \item przetworniki analogowo-cyfrowe: konwersja sygnału analogowego na cyfrowy,
    \item sprzęt komputerowy i oprogramowanie: przetwarzanie informacji. 
\end{itemize}
Systemy DAQ stanowią jedno z głównych narzędzi stosowanych przez naukowców i~inżynierów
na potrzeby testów, pomiarów, a także zadań automatyzacji \cite{data-aq-systems}.
Do zakresu funkcjonalności systemu DAQ można zaliczyć także możliwość utrwalania (składowania)
zebranych danych i/lub informacji na potrzeby przyszłej analizy i przetwarzania. 
System taki można wówczas nazwać \textbf{systemem do zbierania i składowania danych pomiarowych}
(ang. \emph{Data Acquisition And Storage System}, w skrócie: DAS, ASS) \cite{big-data-aq-storage-systems}. 

\subsubsection*{Systemy cybernetyczno-fizyczne}

Poza bierną obserwacją zjawisk fizycznych przydatną może okazać się również
możliwość oddziaływania systemu wirtualnego na rzeczywistość. Ogólny typ systemów
realizujących nazwane funkcjonalności stanowią \textbf{systemy cybernetyczno-fizyczne}
(ang. \emph{Cyber-Physical Systems}, w skrócie: \emph{CPS}).
Są to systemy fizyczne, których funkcjonowanie jest zintegrowane, monitorowane oraz 
nadzorowane przez rdzeń obliczeniowy (system wbudowany). Człon ``cybernetyczny'' odnosi
się do cyfrowych systemów przetwarzania informacji, natomiast człon ``fizyczny'' 
dotyczy stworzonych przez ludzi rzeczywistych systemów technicznych. 
Systemy cybernetyczno-fizyczne stanowią więc skrzyżowanie świata rzeczywistego oraz
wirtualnego (cybernetycznego).
Charakteryzują się one często działaniem w czasie rzeczywistym oraz rozproszeniem 
\cite{iiot-cyber-manufacturing-systems}. 
Obie wymienione cechy zostały dokładniej opisane w rozdziale \ref{isp}, 
jako że mają one szczególne znaczenie w środowisku przemysłowym. \newline

Systemy do zbierania i składowania danych, systemy 
cybernetyczno-fizyczne oraz (ogólnie) systemy informatyczne mogą znaleźć 
zastosowanie w przemyśle. Mogą one wchodzić w skład lub rozszerzać funkcjonalności
projektowanych oraz istniejących systemów, które zostały szczegółowo omówione
w rozdziale \ref{isp}.

\subsection{Systemy informatyczne w przemyśle} \label{isp}

\subsubsection{Podstawowe zagadnienia}

Podrozdział przedstawia podstawowe zagadnienia dotyczące systemów informatyki
stosowanych w przemyśle. Omówiono w nim kolejno: pojęcia związane z procesem przemysłowym,
urządzenia i aparaturę oraz środowiskowe warunki funkcjonowania takich systemów.

\subsubsection*{Proces przemysłowy} 

Do zagadnień podstawowych odnoszących się do środowiska przemysłowego w kontekście 
systemów informatycznych należy zaliczyć pojęcie \textbf{procesu przemysłowego} 
nazywane też \textbf{procesem technologicznym}.
Jest to ciąg określonych zjawisk fizykochemicznych mających na celu wytworzenie
produktu. \textbf{Technologią} nazywamy w tym kontekście uporządkowany opis procesu przemysłowego.
Zbiór powiązanych (współpracujących) procesów przemysłowych nazywa się 
\textbf{procesem produkcyjnym}. Zestaw urządzeń technicznych wykonujących
proces technologiczny nosi miano \textbf{węzła technologicznego}, zaś
uporządkowany przyczynowo-skutkowo zbiór takich węzłów nazywa się \textbf{linią technologiczną}
lub \textbf{linią produkcyjną} \cite{isp}. Działania techniczne i ekonomiczne
składające się na proces produkcyjny obejmuje jednostka organizacyjna zwana
\textbf{jednostką produkcyjną} \cite{jednostka-produkcyjna}. Obecnie w usprawnieniu
działania złożonych jednostek produkcyjnych nieocenionym wsparciem może okazać się
wdrożenie i wykorzystywanie systemów informatycznych.

\subsubsection*{Urządzenia, aparatura} 

W przemyśle wspomaganym przez systemy informatyczne wykorzystywane są 
\textbf{urządzenia aparatury kontrolno-pomiarowej i automatyki} (w skrócie: AKPiA). 
Mogą one stanowić pomost (interfejs, protokół) pomiędzy procesem przemysłowym a systemem informatycznym. 
Zestaw takich urządzeń można określić zbiorem informacji, który nazywamy
\textbf{obiektem przemysłowym} \cite{isp}. Zespoły AKPiA mogą tworzyć opisane
w rozdziale \ref{system-do-zbierania-danych} systemy DAQ oraz CPS. W~obrębie
CPS spotykane jest określenie analogiczne (nieco bardziej ogólne) do obiektu przemysłowego --- 
tzw. \textbf{bliźniaka cyfrowego} (ang. \emph{Digital Twin}). 
Jest to określenie trafne, jako że informacja cyfrowa pod postacią modeli i obiektów
danych reprezentuje niejako drugą (sklonowaną) tożsamość obiektu rzeczywistego.
Dynamiczna natura takich cyfrowych modeli pozwala na wdrażanie innowacyjnych usług,
które obejmują cały cykl życia produktu~\cite{iiot-challenges-opportunities-directions}. 
Ze względu na kierunek przepływu
informacji wśród urządzeń AKPiA można wyróżnić następujące grupy \cite{isp}:
\begin{itemize}
    \item układy \textbf{inicjatorów} (czujniki, sensory, ang. \emph{sensors}) -- rejestrowanie stanu procesu (produkcja informacji),
    \item układy \textbf{wykonawcze} (ang. \emph{actuators}) -- modyfikacja stanu procesu (konsumpcja informacji),
    \item układy \textbf{mieszane} (ang. \emph{mixed}) -- jednoczesna produkcja i konsumpcja informacji.
\end{itemize}

\subsubsection*{Uwarunkowania środowiskowe} 

Omawiając tematykę związaną z systemami przemysłowymi nie sposób pominąć kwestię
warunków otoczenia. Środowisko przemysłowe charakteryzuje się zwiększoną uciążliwością.
Autor pozycji \cite{isp} podkreśla, iż nigdy nie należy zakładać, że środowisko przemysłowe
będzie przyjazne. Wymienia on następujące \textbf{zaburzenia środowiskowe}, które mogą mieć wpływ na pracę
systemu przemysłowego:
\begin{itemize}
    \item otoczenia -- np. temperatura, wilgotność, czynniki chemiczne;
    \item mechaniczne -- np. wstrząsy, zapylenie;
    \item przewodzone -- zaburzenia w sieci zasilania;
    \item elektromagnetyczne -- zaburzenia elektromagnetyczne.
\end{itemize}

\subsubsection{Systemy rozproszone czasu rzeczywistego}

Systemy informatyczne w przemyśle często posiadają cechy \textbf{systemów rozproszonych}.
W systemach tych nie występuje centralne urządzenie
przetwarzające informacje, lecz składają się one z wielu jednostek
przetwarzających o różnym zakresie funkcjonalności. Jest to rozproszenie logiczne,
któremu może (lecz nie musi) towarzyszyć rozproszenie terytorialne. Do zalet
takich systemów zalicza się zwiększoną: moc obliczeniową, niezawodność oraz 
elastyczność (adaptacyjność, rekonfigurowalność)
\cite{isp},
\cite{isp-analiza-przepływu-informacji}. W obrębie systemów przemysłowych
w szczególności stosuje się \textbf{rozproszone systemy sterowania}, 
(ang. \emph{Distributed Control Systems}, w~skrócie: DCS). Do zadań takich systemów
należą sterowanie i wizualizacja procesu przemysłowego. 

Wymaganą cechą procesów przemysłowych jest determinizm \cite{isp}. \textbf{Determinizm}
to twierdzenie, że wszystkie zdarzenia zachodzące w czasie są jednoznacznie
warunkowane przez zdarzenia je poprzedzające. Innymi słowy --- wszystkie zjawiska
podlegają jakimś nieuchronnym prawidłowościom \cite{determinizm}. Ścisła przewidywalność
funkcjonowania systemu jest niezbędna dla zapewnienia pełnej zgodności 
jego działania z założeniami. O ile teoretyczny determinizm zapewniają prawa fizyki,
o tyle w praktyce mogą wystąpić zdarzenia trudne do przewidzenia, na które
system musi być przygotowany \cite{isp}.

Determinizm działania względem czasu określa się mianem \textbf{determinizmu czasowego}.
Postawienie wymagania determinizmu czasowego względem systemu przemysłowego oznacza,
że jego odpowiedź na zdarzenia musi zachodzić w zdefiniowanym, skończonym czasie
\cite{isp}.
Ściślej mówiąc, oczekuje się, że czas obsługi zdarzenia przez system $T_s$ będzie
mniejszy lub równy od określonego czasu granicznego $T_g$, co przedstawia 
wzór \ref{ts-tg}.
\begin{equation}
    T_s \leq T_g\label{ts-tg}
\end{equation}

Systemy charakteryzujące się silnymi ograniczeniami czasowymi nazywamy 
\textbf{systemami czasu rzeczywistego} (ang. \emph{Real Time Systems}, w skrócie: RTS). 
Czas jest w~nich parametrem krytycznym. 
Własności systemów czasu rzeczywistego są typowe dla systemów
nadzorujących przebieg procesów przemysłowych \cite{isp-analiza-przepływu-informacji}.
Ze względu na stopień ograniczeń czasowych wyróżniamy następujące grupy systemów RTS 
\cite{isp}, \cite{isp-analiza-przepływu-informacji}, \cite{iiot-cyber-manufacturing-systems}:
\begin{itemize}
    \item \textbf{systemy z łagodnymi ograniczeniami czasowymi} (ang. \emph{Soft Real-Time Systems}, w~skrócie: SRTS)
     -- systemy, w których przekraczanie ograniczeń czasowych jest tolerowane do pewnego stopnia,
    \item \textbf{systemy z silnymi ograniczeniami czasowymi} (ang. \emph{Firm Real-Time Systems}, w~skrócie: FRTS)
     -- systemy, w których przekroczenie ograniczeń czasowych oznacza tymczasową
     niedostępność oferowanych funkcjonalności, co skutkuje obniżoną jakością realizowanych usług,
    \item \textbf{systemy z ostrymi ograniczeniami czasowymi} (ang. \emph{Hard Real-Time Systems}, w~skrócie: HRTS)
     -- systemy, w których przekroczenie ograniczeń nie jest tolerowane, a ich 
     wystąpienie jest jednoznaczne z awarią systemu. 
\end{itemize}

Systemy posiadające cechy zarówno systemów rozproszonych, jak i systemów
czasu rzeczywistego nazywa się po prostu \textbf{systemami rozproszonymi czasu rzeczywistego}.
W celu umożliwienia współpracy urządzeń w obrębie systemów rozproszonych konieczna
jest \textbf{komunikacja} \cite{isp-analiza-przepływu-informacji}. Jest to zagadnienie kluczowe warunkujące efektywne działanie
systemu --- w szczególności, gdy do stawianych wymagań zaliczymy spełnienie 
ograniczeń czasowych. Rolę medium komunikacyjnego
w systemach komputerowych stanowią \textbf{sieci komputerowe}. Na potrzeby 
przemysłu opracowano dedykowane rozwiązania adresujące opisane problemy.
Przeglądu wybranych przemysłowych sieci komputerowych realizujących wielowarstwowe
modele komunikacyjne dokonano w rozdziale \todo{dać odwołanie do sieci}.

\subsubsection{Dziedzina zastosowań systemów informatycznych w przemyśle}

Jak wcześniej wspomniano, wykorzystanie systemów informatycznych może usprawnić
działania podejmowane w ramach przedsiębiorstwa (należących do niego jednostek produkcyjnych). 
Zespół takich działań można ściślej nazwać \textbf{prowadzeniem procesu}.
Proces ten może mieć dwa aspekty: biznesowy (gospodarczy, zarządczy) oraz techniczny. 
\textbf{Proces biznesowy} to zespół działań dążących do osiągnięcia pewnego celu
zdefiniowanego w ramach przedsiębiorstwa. \textbf{Proces techniczny} dotyczy
obsługi działań związanych z procesem przemysłowym. 


\subsection{Internet Rzeczy}

\subsubsection{Definicje i zastosowania Internetu Rzeczy}

Pojęcie Internetu Rzeczy (ang. \emph{Internet of Things}, w skrócie: IoT) posiada różne definicje. 
Jedną z prostszych jest sieć przedmiotów z wbudowanymi czujnikami, które są podłączone
do Internetu \cite{intro-to-iot}. Bardziej szczegółowe jest ujęcie IoT jako sieci jednoznacznie
identyfikowalnych przedmiotów z wbudowanymi czujnikami, inteligencją obliczeniową
oraz powszechną łącznością z Internetem \cite{iot-hype-to-reality}.
Na nieco wyższym poziomie definiuje się IoT jako globalną infrastrukturę udostępniającą
zaawansowane usługi poprzez połączenie przedmiotów z wykorzystaniem technologii informacyjnych.
\cite{intro-to-iot}. W każdej z przytoczonych
definicji zauważalne jest pewne podobieństwo pomiędzy IoT a pojęciem CPS 
scharakteryzowanym w rozdziale \ref{system-do-zbierania-danych}. Zagadnienia te
posiadają część wspólną, dotyczą podobnej tematyki (przede wszystkim połączenia świata fizycznego i logicznego), 
lecz mają odrębną genezę i uwydatnia się w ich ramach różne aspekty: 
inżynierię systemów sterowania w CPS oraz sieci i komunikację w IoT \cite{cps-vs-iot}.
Głównym zastosowaniem Internetu Rzeczy jest monitorowanie świata rzeczywistego 
oraz wchodzenie z nim w interakcję, co może usprawnić działalność człowieka na wielu płaszczyznach.
Wśród praktycznych zastosowań IoT wymienia się między innymi:
sieci czujników i pomiary rozproszone, urządzenia (gadżety) ubieralne, 
inteligentne budynki, miasta, logistyka, przemysł, a także marketing \cite{internet-reczy}. 
Możliwości, jakie niesie IoT dla przemysłu zostały w sposób szczególny rozpatrzone 
w rozdziale \ref{iiot}.

\subsubsection{``Rzeczy'' w IoT}

Wśród przedmiotów tworzących IoT wyróżnia się dwie kategorie: 
czujniki oraz urządzenia wykonawcze. Są to odpowiedniki aparatury kontrolno-pomiarowej 
wykorzystywanej w przemyśle (patrz: rozdział \ref{isp}). Czujniki umożliwiają
obserwację świata rzeczywistego i jej zapis w postaci cyfrowej.  
Urządzenia wykonawcze są w stanie zmienić stan układu rzeczywistego (wprowadzać w nim ruch)
na podstawie otrzymanej informacji cyfrowej \cite{iot-hype-to-reality}.
Przykładowe czujniki to: przyciski, przełączniki, czujniki ruchu, czujniki gazów,
czujniki wibracji, czujniki temperatury, wilgoci,
nacisku, ultradźwięków i wielu innych wielkości fizycznych. Przykłady urządzeń
wykonawczych to: silniki, siłowniki liniowe, przekaźniki, zawory. Ważnym elementem
infrastruktury IoT są systemy wbudowane typu SoC (ang. \emph{System on a chip}).  
Są to układy elektroniczne, których poszczególne komponenty realizujące wymagane 
funkcjonalności są zintegrowane w ramach jednego, kompletnego układu scalonego. 
System typu SoC mogą zawierać m. in.: procesory lub mikrokontrolery, przetworniki
analogowo-cyfrowe i cyfrowo-analogowe, pamięci, procesory graficzne, interfejsy komunikacyjne \cite{intro-to-iot}, \cite{soc}. 
Obok rozwiązań dedykowanych dużą popularnością cieszą się następujące platformy uniwersalne \cite{intro-to-iot}:

\begin{itemize}
    \item Arduino -- Różne modele płytek rozwojowych wykorzystujące m. in. systemy
    SoC producenta Atmel oparte o mikrokontrolery AVR. Wśród zestawów płytek
    dostępne są moduły rozszerzeń posiadające przewodowe i bezprzewodowe interfejsy sieciowe.
    \item ESP -- Systemy typu SoC producenta Espressif Systems zawierające wbudowane
    bezprzewodowe interfejsy sieciowe.
    \item Raspberry Pi -- Zaawansowane płytki tworzące komputery jednopłtykowe (ang. \emph{single-board computer}).    
    Można na nich uruchomić system operacyjny (również z interfejsem graficznym). Posiadają liczne 
    interfejsy komunikacyjne (w tym sieciowe) \cite{rpi}. 
\end{itemize}

\subsubsection{``Internet'' w IoT}

Wymiana informacji w systemach IoT może odbywać się w sposób przewodowy lub bezprzewodowy.
Stosowane modele komunikacji sieciowej bazują na standardowym modelu OSI bądź
internetowym stosie TCP/IP. Urządzenia Internetu Rzeczy nie zawsze są jednak 
w stanie implementować wszystkie warstwy stosu ze względu na ograniczone zasoby 
(moc obliczeniowa, pamięć, energia elektryczna). W warstwie fizycznej i łącza danych
sieci IoT wykorzystuje się protokoły: Ethernet, WiFi, Bluetooth, sieci komórkowe,
ZigBee, Z-Wave, NFC. W warstwie sieciowej stosuje się: IPv4, IPv6 oraz zoptymalizowaną
pod IoT wersję IPv6 --- 6LoWPAN. Protokoły warstwy transportowej to m. in.: TCP i UDP. 
Warstwy sesji, prezentacji i aplikacji implementują powszechne internetowe protokoły, jak
HTTP, RTP, SMTP. Na szczególną uwagę zasługują protokoły najwyższych warstw zaprojektowanie
z myślą o IoT, są to m.in. MQTT i CoAP \cite{internet-reczy}, \cite{intro-to-iot}, \cite{iot-hype-to-reality}.

\subsubsection*{Protokół MQTT}

MQTT (ang. \emph{Message Queue Telemetry Transport}) to lekki pod względem wykorzystania zasobów
protokół polegający na wymianie komunikatów (ang. \emph{messages}). Architektura komunikacji
jest oparty na wzorcu publikuj-subskrybuj (ang. \emph{publish-subscribe}).
Klienci łączą się do centralnego serwera (pośrednika, ang. \emph{broker}), za pośrednictwem
którego mogą wysyłać i odbierać komunikaty pod konkretnym adresem (tematem, ang. \emph{topic}).
Tematy tworzą hierarchiczną strukturę analogiczną do URL. MQTT jest protokołem binarnym
wykorzystującym TCP w warstwie transportowej \cite{iot-hype-to-reality}.

\subsubsection*{Protokół CoAP}

CoAP (ang. \emph{Constrained Application Protocol}) to zoptymalizowana alternatywa dla HTTP
przeznaczona dla urządzeń o ograniczonych zasobach. Posiada Wykorzystuje m. in. protokół
UDP w warstwie transportowej, krótsze nagłówki w formie binarnej. 
CoAP wywodzi się ze wzorca architektonicznego REST (ang. \emph{Representational state transfer}), 
jest więc zorientowany na zasoby umożliwiając ich tworzenie, odczyt, aktualizację
i usuwanie (ang. /emph{Create, Read, Update, Delete}, w skrócie: CRUD). 
Identycznie jak w przypadku HTTP, architektura komunikacji protokołu CoAP to klient-serwer 
\cite{intro-to-iot}, \cite{iot-hype-to-reality}.

\subsubsection*{Modele komunikacyjne}

W sieciach IoT rozróżnia się trzy następujące modele komunikacyjne \cite{intro-to-iot}:

\begin{itemize}
    \item Urządzenie -- Urządzenie (ang. \emph{Machine to Machine, Device to Device}, w skrócie: M2M)
    --- Urządzenia komunikują się bezpośrednio bez potrzeby translacji czy wykonywania 
    skomplikowanego przetwarzania danych (dotyczy to zarówno urządzeń tej samej klasy, jak i różnej);
    \item Urządzenie -- Brama (ang. \emph{Device to Gateway})
    --- Występuje, gdy zachodzi potrzeba translacji informacji wymienianej pomiędzy różnymi sieciami.
    Translacją zajmuje brama (ang. \emph{gateway}), którą tworzy dedykowane oprogramowanie i/lub urządzenie;
    \item Urządzenie -- Chmura (ang. \emph{Device to Cloud})
    --- Występuje, gdy zachodzi potrzeba zaawansowanego przetwarzania danych 
    (statystyka, eksploracja danych, big data, składowanie, itp.), 
    co nie jest możliwe w przypadku urządzeń o ograniczonych zasobach obliczeniowych, 
    pamięciowych i energetycznych. Wówczas gromadzone przez urządzenia dane są 
    przesyłane do platform opartych o usługi chmurowe. Może to być zarówno chmura publiczna 
    (np. AWS, Azure), jak i lokalna, prywatna infrastruktura informatyczna. 
    Niektóre urządzenia są w stanie komunikować się z chmurą bezpośrednio, inne
    korzystają w tym celu z bram.
\end{itemize}

\subsubsection{Usługi chmurowe}

Do niedawna większość przedsiębiorstw chcących czerpać korzyści z wykorzystania
systemów informatycznych było zmuszonych do inwestowania we własną infrastrukturę
komputerową. Przetwarzanie w chmurze (ang. \emph{cloud computing}) pozwala obecnie 
na migrację części lub całości infrastruktury informatycznej na dostawców chmury,
takich jak Microsoft Azure, Amazon AWS, Google Cloud, czy IBM Cloud. Usługi chmurowe
udostępnianie przez wymienionych i innych dostawców noszą miano chmury publicznej. 
Są one dostępne na życzenie za pośrednictwem Internetu i posiadają elastyczny
system rozliczeń, w którym koszta są zależne od faktycznie zużytych zasobów
(cykle procesorów, pamięć, wykorzystanie łącza, liczba zapytań). Do zalet takiego
rozwiązania należą: łatwa skalowalność, konfigurowalność i duży wybór usług
ułatwiających utrzymanie infrastruktury informatycznej. Spośród wad należy wymienić
zależność od firm trzecich (dostawców chmury) oraz brak pełnej kontroli nad 
posiadaną infrastrukturą \cite{iot-hype-to-reality}.

W ramach usług dostawcy chmury udostępniają m.in.: przetwarzanie danych, 
analitykę, eksplorację danych, uczenie maszynowe, konteneryzację, relacyjne i nierelacyjne
bazy danych, integrację aplikacji, hosting aplikacji i stron internetowych, 
maszyny wirtualne, narzędzia programistyczne, zarządzanie użytkwnikami, a także
platformy do zarządzania systemami IoT \cite{aws}, \cite{azure}.

Zarządzanie dużą ilością danych jest podstawowym zadaniem systemu IoT. Składa się ono
na: zbieranie, filtrowanie, agregację, przetwarzanie, składowanie, udostępnianie,
wizualizację oraz zabezpieczanie \cite{intro-to-iot}. Usługi chmurowe umożliwiają złożone zarządzanie
danymi oferując jednocześnie skalowalność, zdalny dostęp, opłaty zależne od zapotrzebowania
i gotowe rozwiązania programistyczne skracające czas rozwoju systemów. 
Wymienione usługi, jak i dedykowane chmurowe platformy do administrowania 
urządzeniami mogą nieść wiele korzyści dla projektowanego lub rozwijanego
systemu IoT \cite{measuring-value-of-cloud-computing}, \cite{iot-and-cloud}, \cite{iot-in-industrial-sector}. 

\subsection{Internet rzeczy w przemyśle}\label{iiot}

\subsection{Zakres tematyczny realizowanego projektu}


\section{Wymagania i narzędzia}

\section{Specyfikacja zewnętrzna}

\section{Specyfikacja wewnętrzna}

\section{Weryfikacja i walidacja}

\section{Podsumowanie i wnioski}

\printbibliography

\end{document}