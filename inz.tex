\documentclass[a4paper, 12pt, twoside]{article}
\usepackage[a4paper, top=2.5cm, bottom=2.5cm, left=2.5cm, right=2.5 cm, bindingoffset=0.5cm]{geometry}
\usepackage[onehalfspacing]{setspace}
\usepackage[utf8]{inputenc}
\usepackage[polish]{babel}
\usepackage[T1]{fontenc}
\usepackage{graphicx}
\usepackage{ragged2e}
\usepackage{indentfirst}
\usepackage{csquotes}
\usepackage[backend=biber, maxbibnames=30]{biblatex}
\usepackage{amsmath}
\usepackage{todonotes}
\usepackage{float} % to use figure positioning [H]
\usepackage{multirow} % tabele wielowierszowe komórki
\usepackage{array} % jw
\usepackage{listings}
\lstset
{ %Formatting for code in appendix
basicstyle=\ttfamily\scriptsize,    
numbers=left,
    stepnumber=1,
    showstringspaces=false,
    frame=single,
    tabsize=1,
    breaklines=true,
    breakatwhitespace=false,
    xleftmargin=2.5cm,
    xrightmargin=1.5cm
}
\addbibresource{inz.bib}
\graphicspath{ {./images/} }

\title{System IIoT do zbierania danych pomiarowych}
\date{}

\begin{document}

\clearpage\maketitle %no page number on title page
\thispagestyle{empty}

\listoftodos

\newpage

\tableofcontents

\newpage
\section{Wstęp}\label{intro}

Systemy informatyczne znajdują obecnie szerokie zastosowanie w wielu przedsiębiorstwach.
Do zadań takich systemów należą przede wszystkim: automatyzacja procesów,
ułatwienie zarządzania zasobami (ludzkimi, materialnymi, finansowymi, dokumentacją, wiedzą),
gromadzenie informacji na temat prowadzonej
działalności na potrzeby raportowania, nadzoru, optymalizacji i rozwoju strategii, a także
usprawnienie komunikacji z klientami i kontrahentami. Korzyści wynikające z wdrożenia
rozwiązań informatycznych wykorzystuje się również w środowisku
przemysłowym. Obok wymienionych zadań (obejmujących szerszy zakres działalności
przedsiębiorstwa) podstawową rolą przemysłowych systemów informatycznych
jest wspomaganie prowadzenia procesów przemysłowych (produkcji przemysłowej).
Istotnym aspektem działania takich systemów
jest zbieranie danych pomiarowych. Dane gromadzone przez zainstalowaną w systemie produkcji
aparaturę pomiarową, jak i wprowadzane przez ludzi lub inne systemy, determinują działanie układów automatyki,
umożliwiają nadzór, diagnostykę, regulację, monitorowanie i wizualizację procesów przemysłowych.
Ponadto stosunkowo duża liczba gromadzonych danych poddana zaawansowanemu przetwarzaniu
(ang. \emph{big data})
może dostarczyć cennej wiedzy w kontekście prowadzenia biznesu.
Wymaganiem charakterystycznym stawianym wobec systemów informatyki przemysłowej
jest konieczność działania w~ramach ograniczeń nakładanych przez specyficzne środowisko.
Należą do nich: determinizm działań i determinizm czasowy, niezawodne funkcjonowanie (odporne
na często niesprzyjające warunki), bezpieczeństwo danych i~urządzeń~\cite{isp}.

Zbieranie znacznej ilości danych w celu ich dalszego przetwarzania wymienia się
wśród głównych zadań Internetu Rzeczy (ang. \emph{Internet of Things}, w skrócie:~IoT).
Jest to koncepcja, w myśl której przedmioty wyposażone w czujniki i/lub elementy
wykonawcze są podłączane do Internetu. Systemy wykorzystujące Internet Rzeczy stały się szczególnie popularne
w ostatnich latach dzięki czynnikom ułatwiającym ich realizację:
znaczący rozwój systemów komputerowych małej i~średniej skali oraz urządzeń mobilnych, upowszechniony
dostęp do Internetu, stopniowe wdrażanie protokołu IPv6, duże postępy w dziedzinie
eksploracji danych, uczenia maszynowego i inteligencji obliczeniowej,
a~także dynamiczny rozwój usług opartych na chmurze obliczeniowej~\cite{intro-to-iot}.
Dziedzina zastosowań IoT jest szeroka, gdyż obejmuje ona urządzenia
o wielu różnych przeznaczeniach. Powszechne stało się używanie określenia ,,inteligenty''
(ang. \emph{smart}) wobec tego typu przedmiotów lub środowisk, w których są one używane.  W~ten sposób
powstały koncepcje inteligentnych gadżetów i urządzeń gospodarstwa domowego,
rolnictwa, logistyki, systemów opieki zdrowotnej, budynków, a nawet miast.
Do obszarów wykorzystania IoT zalicza się także przemysł. Koncepcja ta nosi
nazwę Przemysłowego Internetu Rzeczy (ang. \emph{Industrial Internet of Things}, w skrócie IIoT).
Głównym przeznaczeniem IIoT jest gromadzenie wiedzy na temat procesu przemysłowego,
a~w~konsekwencji --- zwiększanie jego efektywności \cite{iiot-challenges-opportunities-directions}.

Celem niniejszej pracy jest zaprojektowanie systemu informatycznego
wraz z prototypem, którego podstawowym zadaniem jest zbieranie i składowanie danych pomiarowych.
Obszarem zastosowań systemu jest środowisko przemysłowe.
W realizacji projektu kładzie się nacisk na wykorzystanie technologii IoT
oraz chmury obliczeniowej. Źródłem danych w systemie
jest aparatura pomiarowa (czujniki) funkcjonująca w~ramach systemów automatyki i IoT,
inne systemy informatyczne wdrożone w przedsiębiorstwie,
a także pracownicy obsługujący proces przemysłowy (mogą oni monitorować jego działanie
i~wprowadzać wyniki obserwacji do systemu, jak również wyzwalać określone zdarzenia).
Przedmiotem pracy jest ogólne opracowanie architektury
systemu adresujące przedstawiony problem aplikacyjny. Stąd opisane w kolejnych rozdziałach
rozwiązanie teoretyczne abstrahuje od szczegółowych przpyadków użycia, konkretnych urządzeń
technicznych czy technologii. Oczekiwaną cechą systemu jest możliwość dostosowania
do różnych konfiguracji sprzętu i oprogramowania stosowanych w~IoT i~dedykowanych
rozwiązaniach przemysłowych. Jedną z~możliwych implementacji i~jednocześnie weryfikacją
zaproponowanej koncepcji systemu jest wykonany prototyp. Na jego potrzeby zdefiniowano
kilka ściślej sprecyzowanych scenariuszy zastosowania, dokonano wyboru urządzeń
i określonych usług chmurowych, jak również bibliotek i narzędzi programistycznych.

W rozdziale \ref{analiza} przedstawiono przegląd dostępnej literatury związanej z podjętą problematyką.
W pierwszej kolejności omówiono z osobna poszczególne zagadnienia, które
obejmuje temat pracy, przybliżając w ten sposób jego dziedzinę.
Następnie zebrane informacje zestawiono w kontekście realizowanego zadania projektowego,
wskazano możliwe problemy i potencjalne rozwiązania wraz z zarysem architektury
systemu. Opisano również kilka dostępnych na rynku rozwiązań komercyjnych,
które adresują podobny zakres funkcjonalny. W rozdziale \ref{wymagania} sprecyzowano
wymagania funkcjonalne i niefunkcjonalne, które ma realizować projektowany system
oraz zdefiniowano ogólne przypadki użycia i przykładowe scenariusze wykorzystania systemu
implementowane w ramach prototypu. Rozdział \ref{spec-wew} zawiera
szczegółowy opis architektury zaprojektowanego systemu i jego implementacji
w postaci prototypu. Przedstawiono poszczególne
komponenty systemu oraz~omówiono ich funkcje i sposoby wzajemnej interakcji.
W rozdziale \ref{spec-zew} zamieszczono informacje dotyczące
wdrażania i~używania systemu, a także wymagań sprzętowych i programowych.
Poruszono również kwestie bezpieczeństwa. Rozdział \ref{testy} zawiera
opis stosowanej metodyki testowania i przypadków testowych oraz przedstawia wyniki przeprowadzonych
testów z uwzględnieniem napotkanych problemów. W rozdziale \ref{wnioski} zamieszczono
podsumowanie i wnioski, które nasunęły się po zakończeniu prac nad projektem.
Uzyskany rezultat zestawiono z postawionymi celami i określonymi wymaganiami.
Rozważono również kierunki dalszego rozwoju systemu.

\newpage
\section{Analiza tematu}\label{analiza}

W niniejszym rozdziale przedstawiono rozważania teoretyczne dotyczące projektowanego
systemu opierające się na analizie dostępnej literatury. Realizowany temat
dotyczy zasadniczo czterech zagadnień: systemów do zbierania danych pomiarowych,
systemów informatycznych w przemyśle, IoT i IIoT.  Do każdego z nich odnoszą się
odpowiednio kolejne podrozdziały przedstawiające podstawowe pojęcia, problemy i~rozwiązania
w danej dziedzinie.
W ostatnim podrozdziale dokonano zestawienia zebranych wiadomości w kontekście
celu pracy.

\subsection{Systemy informatyczne do zbierania danych pomiarowych} \label{system-do-zbierania-danych}

Zgodnie z encyklopedią \textbf{system informatyczny} to zespół systemów komputerowych,
sieci oraz oprogramowania, których celem zastosowania jest
przetwarzanie informacji \cite{system-informatyczny}.
\textbf{Przetwarzaniem informacji} nazywa się proces akwizycji, utrwalania, udostępniania,
organizacji, interpretacji i wizualizacji informacji \cite{information-science},
\cite{information-processing}. Warto w~tym miejscu doprecyzować pojęcia
\textbf{informacji} oraz \textbf{danych}. Dane są pewną skodyfikowaną reprezentacją
faktów. Natomiast dane, które podlegają intepretacji i nabierają znaczenia
stają się informacją \cite{dane-informacja-wiedza}.

Jednym z możliwych źródeł informacji w systemie informatycznym mogą być
\textbf{systemy do zbierania danych pomiarowych} (system akwizycji danych, ang. \emph{Data Acquisition Systems}, w skrócie: DAQ).
Są to systemy, których rolą jest gromadzenie w formie cyfrowej danych opisujących
zjawiska fizyczne. Systemy DAQ tworzą między innymi następujące komponenty
o odpowiednich przeznaczeniach:
\begin{itemize}
      \itemsep0em
      \item czujniki (sensory, ang. \emph{sensors}) --- przetwarzanie zjawisk fizycznych na analogowy sygnał elektryczny,
      \item przetworniki analogowo-cyfrowe --- konwersja sygnału analogowego na cyfrowy,
      \item sprzęt komputerowy i oprogramowanie --- przetwarzanie informacji.
\end{itemize}
Systemy DAQ stanowią jedno z głównych narzędzi stosowanych przez naukowców i~inżynierów
na potrzeby testów, pomiarów, a także zadań automatyzacji \cite{data-aq-systems}.

\subsection{Systemy informatyczne w przemyśle} \label{isp}

W podrozdziale podjęto zagadnienie systemów informatycznych wdrażanych w środowisku przemysłowym.
Omówiono kolejno: pojęcia podstawowe, fundamentalne informacje dotyczące systemów rozproszonych i czasu rzeczywistego
oraz zaprezentowano przykładową architekturę systemów przemysłowych.

\subsubsection{Podstawowe zagadnienia}

Podstawowym zagadnieniem odnoszącym się do środowsika przemysłowego jest pojęcie
\textbf{procesu przemysłowego} nazywane też \textbf{procesem technologicznym}.
Jest to ciąg określonych zjawisk fizykochemicznych mających na celu wytworzenie
produktu. Dla zwiększenia efektywności, bezpieczeństwa i kontroli nad procesami
przemysłowymi praktykuje się wspomaganie ich systemami informatycznymi \cite{isp}.

W dziedzinie informatyki przeymsłowej wykorzystuje się tzw.
\textbf{urządzenia aparatury kontrolno-pomiarowej i automatyki} (w skrócie: AKPiA).
Stanowią one pomost pomiędzy procesem przemysłowym a systemem informatycznym.
Wśród urządzeń AKPiA można wyróżnić następujące grupy \cite{isp}:
\begin{itemize}
      \itemsep0em
      \item układy \textbf{inicjatorów} (czujniki, sensory, ang. \emph{sensors}) --- rejestrowanie stanu procesu,
      \item układy \textbf{wykonawcze} (ang. \emph{actuators}) --- modyfikacja stanu procesu,
      \item układy \textbf{mieszane} (ang. \emph{mixed}) --- jednoczesna rejestracja i modyfikacja stanu procesu.
\end{itemize}

\noindent Zestaw urządzeń AKPiA można zamodelować zbiorem informacji, który nazywa się
\textbf{obiektem przemysłowym} \cite{isp}.

W ostatnich latach na popularności zyskał pomysł wykorzystania
w przemyśle systemów cybernetyczno-fizycznych
(ang. \emph{Cyber-Physical Systems}). Koncepcja ta polega na osadzaniu
komputerów o niewielkich rozmiarach w zwykłych obiektach fizycznych, a w kontekście
przemysłowym --- w elementach obsługujących procesy przemysłowe. Mówi się
wówczas o tzw. inteligencji wbudowanej. W~obrębie
CPS spotykane jest określenie analogiczne (nieco bardziej ogólne) do obiektu przemysłowego ---
tzw. \textbf{bliźniaka cyfrowego} (ang. \emph{Digital Twin}). Jest to
cyfrowe odwzorowanie cech i atrybutów rzeczywistego obiektu \cite{isp}, \cite{iiot-challenges-opportunities-directions}.


Omawiając tematykę związaną z systemami przemysłowymi nie sposób pominąć kwestię
warunków otoczenia. Środowisko przemysłowe charakteryzuje się zwiększoną uciążliwością.
Wymienia się następujące \textbf{zaburzenia środowiskowe}, które mogą mieć wpływ na pracę
systemu przemysłowego: zaburzenia otoczenia (np. temperatura, wilgotność, czynniki chemiczne),
zaburzenia mechaniczne, przewodzone i elektromagnetyczne \cite{isp}.

\subsubsection{Systemy rozproszone czasu rzeczywistego}

Systemy informatyczne w przemyśle często posiadają cechy \textbf{systemów rozproszonych}.
W systemach tych nie występuje centralne urządzenie
przetwarzające informacje, lecz składają się one z wielu jednostek
przetwarzających o różnym zakresie funkcjonalności. Do zalet
takich systemów zalicza się zwiększoną moc obliczeniową względem systemów nierozproszonych,
niezawodność oraz elastyczność (adaptacyjność, rekonfigurowalność)
\cite{isp},
\cite{isp-analiza-przepływu-informacji}. W obrębie systemów przemysłowych
w szczególności stosuje się \textbf{rozproszone systemy sterowania},
(ang. \emph{Distributed Control Systems}, w~skrócie: DCS). Do zadań takich systemów
należą sterowanie i wizualizacja procesu przemysłowego.

Wymaganą cechą procesów przemysłowych jest \textbf{determinizm czasowy}.
Postawienie wymagania determinizmu czasowego względem systemu przemysłowego oznacza,
że jego odpowiedź na zdarzenia musi zachodzić zawsze w zdefiniowanym, skończonym czasie
\cite{isp}.

Systemy charakteryzujące się m. in. determinizmem czasowym nazywa się
\textbf{systemami czasu rzeczywistego} (ang. \emph{Real Time Systems}, w skrócie: RTS).
Systemy posiadające cechy zarówno systemów rozproszonych, jak i systemów
czasu rzeczywistego nazywa się \textbf{systemami rozproszonymi czasu rzeczywistego}.
Zagadnieniem kluczowym w kontekście takich systemów
jest \textbf{komunikacja}, którą umożliwiają sieci i protokoły.
Na potrzeby przemysłu opracowano rozwiązania odpowiadające również na potrzebę
działania w ścisłych ograniczeniach czasowych. Do przykładowych sieci należą przede wszystkim
sieci polowe (ang. \emph{fieldbus}) i Ethernet czasu rzeczywistego (Ethernet przemysłowy, ang. \emph{Real-Time Ethernet}, w skrócie: RTE).
Sieci RTE posiadają obecnie wiele technicznych realizacji. Krokiem w kierunku
standaryzacji  jest koncept sieci TSN (ang. \emph{Time-Sensitive Networks})
rozwijany od ostatniej dekady w ramach specyfikacji sieci Ethernet należących do grupy IEEE 802.
Dedykowane dla sieci przemysłowych protokoły to m. in. Profibus/Profinet,
Modbus, EtherCAT, Ethernet Powerlink i inne \cite{isp}, \cite{isp-analiza-przepływu-informacji},
\cite{tsn-for-dummies}, \cite{rte-standards-and-impl}.



\subsubsection{Architektura systemów informatycznych w przemyśle}\label{arch-przemysl}

Dziedzina funkcjonalna systemów informatycznych w przemyśle jest stosunkowo obszerna.
Są one w stanie wspomagać zarówno przebieg procesu przemysłowego,
jak i~kompleksowo usprawniać procesy funkcjonujące na wyższych poziomach przedsiębiorstwa
--- np. te związane z zarządzaniem zasobami. W literaturze spotkać można różne
modele przedstawiające zakres działania przemysłowych systemów informatycznych.
Swego czasu popularny był tzw. model piramidowy (warstwowy, hierarchiczny)
przedstawiony na rys. \ref{fig:piramida}. Został on wyparty m. in.
przez model RAMI 4.0, który opisano w rozdziale \ref{iiot} w kontekście IIoT.
W modelu piramidowym wyróżnia się następujące warstwy:
\cite{systemy-automatyki-przemyslowej}, \cite{isp}, \cite{iiot-cyber-manufacturing-systems}, \cite{intro-to-iot}:
\begin{itemize}
      \itemsep0em
      \item Warstwa produkcyjna --- tworzą ją urządzenia automatyki przemysłowej
            oraz aparatury kontrolno-pomiarowej wraz z łączącymi je sieciami komputerowymi,
      \item Warstwa operacyjna --- należą do niej systemy realizacji produkcji
            (ang. \emph{Manufacturing Execution System}, w skrócie: MES) oraz systemy
            zbierania danych i~sprawowania kontroli SCADA/HMI
            (ang. \emph{Supervisory Control and Data Acquisition, Human Machine Interface}),
      \item Warstwa biznesowa --- należą do niej wysokopoziomowe systemy planowania
            zasobów przedsiębiorstwa ERP (ang. \emph{Enterprise Resource Planning}),
            systemy analizy i produkcji SAP (ang. \emph{Systems Analytics and Product}),
            systemy zarządzania relacjami z klientami CRM (ang.\emph{Customer Relationship Management})
            oraz wiele innych.
\end{itemize}

\begin{figure}[h]
      \centering
      \includegraphics[width=0.55\textwidth]{piramida.png}
      \caption{Model piramidowy informatycznych systemów przemysłowych.}
      \label{fig:piramida}
\end{figure}

\subsection{Internet Rzeczy}\label{iot}

W podrozdziale przedstawiono przegląd definicji, zastosowań, a także
elementów tworzących IoT. Podrozdział zawiera również odniesienie do ważnego
w~kontekście omawianego zagadnienia przetwarzania w chmurze.

\subsubsection{Definicje i zastosowania Internetu Rzeczy}

\textbf{Internet Rzeczy} (ang. \emph{Internet of Things}, w skrócie: IoT) posiada różne definicje.
Jedną z podstawowych jest: sieć przedmiotów z wbudowanymi czujnikami, które są podłączone
do Internetu \cite{intro-to-iot}. Na nieco wyższym poziomie definiuje się IoT jako globalną infrastrukturę udostępniającą
zaawansowane usługi poprzez połączenie przedmiotów z wykorzystaniem technologii informacyjnych.
\cite{intro-to-iot}.

Głównym zastosowaniem Internetu Rzeczy jest monitorowanie świata rzeczywistego
oraz wchodzenie z nim w interakcję, co może usprawnić działalność człowieka na wielu płaszczyznach.
Wśród praktycznych zastosowań IoT wymienia się między innymi:
sieci czujników i pomiary rozproszone --- w szczególności z wykorzystaniem
technologii bezprzewodowej sieci czujników
(ang. \emph{Wireless Sensor Networks}, w skrócie: WSN) lub też
bezprzewodowej sieci czujników i urządzeń wykonawczych
(ang. \emph{Wireless Sensor and Actuator Networks}, w skrócie: WSAN), urządzenia (gadżety) ubieralne,
inteligentne budynki, miasta, logistyka, przemysł, a także marketing \cite{internet-reczy},
\cite{wsan}.
Możliwości, jakie niesie IoT dla przemysłu zostały w sposób szczególny rozpatrzone
w rozdziale \ref{iiot}.

\subsubsection{,,Rzeczy'' w IoT}

Wśród przedmiotów tworzących IoT wyróżnia się dwie kategorie:
\textbf{czujniki oraz urządzenia wykonawcze}. Są to odpowiedniki aparatury kontrolno-pomiarowej
wykorzystywanej w przemyśle (patrz: rozdział \ref{isp}). Czujniki umożliwiają
obserwację świata rzeczywistego i jej zapis w postaci cyfrowej.
Urządzenia wykonawcze są w stanie zmienić stan układu rzeczywistego
na podstawie otrzymanej informacji \cite{iot-hype-to-reality}.
Przykładowe czujniki to: przyciski, przełączniki, czujniki ruchu, czujniki gazów,
czujniki wibracji, czujniki temperatury, wilgoci,
nacisku, ultradźwięków i wielu innych wielkości fizycznych. Przykłady urządzeń
wykonawczych to: silniki, siłowniki liniowe, przekaźniki, zawory.

Ważnym elementem
infrastruktury IoT są systemy wbudowane typu SoC (ang.~\emph{System on a Chip}).
Są to układy elektroniczne, których poszczególne komponenty realizujące wymagane
funkcjonalności są zintegrowane w ramach jednego, kompletnego układu scalonego.
System typu SoC mogą zawierać m. in.: procesory lub mikrokontrolery, przetworniki
analogowo-cyfrowe i cyfrowo-analogowe, pamięci, procesory graficzne, interfejsy komunikacyjne \cite{intro-to-iot}, \cite{soc}.
Obok rozwiązań dedykowanych dużą popularnością cieszą się następujące platformy uniwersalne \cite{intro-to-iot},
\cite{rpi}, \cite{stm32}, \cite{mikrokontrolery-stm32}, \cite{beaglebone}, \cite{beaglebone-rtu}, \cite{beaglebone-vs-rpi}:
\begin{itemize}
      \itemsep0em
      \item Arduino --- Różne modele płytek rozwojowych wykorzystujące m. in. systemy
            SoC producenta Atmel oparte o mikrokontrolery AVR. Wśród zestawów płytek
            dostępne są moduły rozszerzeń posiadające przewodowe i bezprzewodowe interfejsy sieciowe;
      \item ESP --- Systemy typu SoC producenta Espressif Systems zawierające m. in. wbudowane
            bezprzewodowe interfejsy sieciowe;
      \item Raspberry Pi --- Zaawansowane płytki tworzące komputery jednopłytkowe (ang. \emph{single-board computers})
            wykorzystujące m. in. procesory ARM Cortex.
            Można na nich uruchomić system operacyjny (również z interfejsem graficznym). Posiadają liczne
            interfejsy komunikacyjne (w tym sieciowe);
      \item STM32 --- Rodzina 32-bitowych mikrokontrolerów opartych na procesorach
            ARM Cortex. Platformę cechuje stosunkowo duża moc obliczeniowa, możliwość działania w trybie niskiego poboru energii
            oraz ograniczeniach czasowych;
      \item BeagleBone --- Komputery jednopłytkowe podobne do RaspberryPi
            oparte na procesorach ARM. Można na nich uruchomić system operacyjny.
            Cechą wyróżniającą jest możliwość funkcjonowania w czasie rzeczywistym
            z wykorzystaniem programowalnej jednostki czasu rzeczywistego
            (ang. \emph{Programmable Real-Time Unit}, w skrócie PRU).
\end{itemize}
Wymienione platformy uniwersalne mogą stanowić pomost pomiędzy główną siecią w systemie
IoT (np. Ethernet lub WiFi) a innymi urządzeniami
peryferyjnymi komunikującymi się za pośrednictwem
uniwersalnych złącz pinowych GPIO albo interfejsów szeregowych, takich jak: SPI, I2C, RS232, USB \cite{intro-to-iot}.

\subsubsection{,,Internet'' w IoT}

Wymiana informacji w systemach IoT może odbywać się w sposób przewodowy lub bezprzewodowy.
Stosowane modele komunikacji sieciowej bazują na standardowym modelu OSI bądź
internetowym stosie TCP/IP. W warstwie fizycznej i łącza danych
sieci~IoT wykorzystuje się: Ethernet, WiFi, Bluetooth, sieci komórkowe,
ZigBee, Z-Wave, NFC. W warstwie sieciowej stosuje się protokoły: IPv4, IPv6 oraz zoptymalizowaną
pod IoT wersję IPv6 --- 6LoWPAN. Protokoły warstwy transportowej to m.~in. TCP i UDP.
Warstwy sesji, prezentacji i aplikacji implementują powszechne używane internetowe protokoły, np.
HTTP, RTP, SMTP. Wśród nich opracowano również
protokoły dedykowane dla IoT --- są to m.~in. MQTT i CoAP \cite{internet-reczy}, \cite{intro-to-iot}, \cite{iot-hype-to-reality}.


W sieciach IoT rozróżnia się trzy następujące modele komunikacyjne \cite{intro-to-iot}:

\begin{itemize}
      \itemsep0em
      \item Urządzenie -- Urządzenie (ang. \emph{Machine to Machine, Device to Device}, w skrócie:~M2M)
            --- Urządzenia komunikują się pomiędzy sobą bez potrzeby translacji czy wykonywania
            skomplikowanego przetwarzania danych.
      \item Urządzenie -- Brama (ang. \emph{Device to Gateway})
            --- Występuje, gdy zachodzi potrzeba translacji informacji wymienianej pomiędzy różnymi sieciami.
            Translacją zajmuje brama (ang. \emph{gateway}), którą tworzy dedykowane oprogramowanie i/lub urządzenie;
      \item Urządzenie -- Chmura (ang. \emph{Device to Cloud})
            --- Występuje, gdy zachodzi potrzeba zaawansowanego przetwarzania danych
            z wykorzystaniem chmury (statystyka, eksploracja danych, big data, składowanie, itp.).
\end{itemize}

\subsubsection{Usługi chmurowe}

\textbf{Przetwarzanie w chmurze} (ang. \emph{cloud computing}) pozwala
na migrację części lub całości infrastruktury informatycznej do tzw. chmury.
Wśród popularnych rozwiązań znajdują się:
Microsoft Azure, Amazon AWS, Google Cloud, czy IBM Cloud. Usługi chmurowe
udostępnianie przez wymienionych i wielu innych dostawców za pośrednictwem Internetu
noszą miano \textbf{chmury publicznej}.
Są one dostępne na życzenie i posiadają elastyczny
system rozliczeń, w którym koszta są zależne od faktycznie zużytych zasobów informatycznych
(cykle procesorów, pamięć, wykorzystanie łącza, liczba zapytań). Do zalet takiego
rozwiązania należą: łatwa skalowalność, konfigurowalność i duży wybór usług
ułatwiających utrzymanie infrastruktury informatycznej. Spośród wad należy wymienić
zależność od firm trzecich (dostawców chmury) oraz brak pełnej kontroli nad
posiadaną infrastrukturą. Niektóre firmy rozwijają też podobne, lecz dostępne
wyłącznie lokalnie, rozbudowane platformy infrastruktury informatycznej, zwane \textbf{chmurą prywatną}.
Tego typu rozwiązania są kosztowne, lecz dają pełną kontrolę nad posiadanymi zasobami \cite{iot-hype-to-reality}.

W ramach swoich usług dostawcy chmury udostępniają m.in.: przetwarzanie danych,
analitykę, eksplorację danych, uczenie maszynowe, konteneryzację, relacyjne i~nierelacyjne
bazy danych, integrację aplikacji, hosting aplikacji i stron internetowych,
maszyny wirtualne, narzędzia programistyczne, zarządzanie użytkownikami, a~ponadto
platformy do zarządzania systemami IoT \cite{aws}, \cite{azure}.

Zarządzanie dużą ilością danych jest podstawowym zadaniem systemów IoT.
Usługi chmurowe umożliwiają zaawansowane przetwarzanie danych
oferując jednocześnie skalowalność, zdalny dostęp, opłaty zależne od zapotrzebowania
i gotowe rozwiązania programistyczne skracające czas rozwoju systemów
\cite{intro-to-iot}, \cite{measuring-value-of-cloud-computing}, \cite{iot-and-cloud}, \cite{iot-in-industrial-sector}.

\subsection{Internet rzeczy w przemyśle}\label{iiot}

\textbf{Przemysłowym Internetem Rzeczy} (ang. \emph{Industrial Internet of Things}, w~skrócie: IIoT)
nazywa się rozwiązania IoT znajdujące zastosowanie w przemyśle.
IIoT ma umożliwić lepsze zrozumienie procesów
przemysłowych, a w związku z tym zapewnić większą wydajność i zrównoważenie
produkcji. Innymi słowy, IIoT ma za zadanie ułatwić integrację \textbf{technologii operacyjnych}
(ang. \emph{Operation Technologies}, w skrócie: OT) oraz
\textbf{technologii informatycznych} (ang. \emph{Information Technologies}, w skrócie: IT)
\cite{iiot-challenges-opportunities-directions}.
Technologie operacyjne to systemy nadzorujące przebieg procesów przemysłowych,
zaś technologie informatyczne to systemy, których rolą jest jest gromadzenie
i przetwarzanie informacji wartościowych z perspektywy przedsiębiorstwa
\cite{ot-it-categorization-of-customer-concerns}.

Ze względu na konieczność funkcjonowania w specyficznym środowisku przemysłowym
IIoT posiada pewne cechy odróżniające od standardowego (konsumenckiego) IoT.
Wśród nich należy wymienić \cite{iiot-challenges-opportunities-directions}:
\begin{itemize}
      \itemsep0em
      \item Rozwój o charakterze ewolucyjnym, nie rewolucyjnym
            --- Wdrażając systemy IIoT należy się liczyć z dużą bezwładnością
            przemysłowego zaplecza technicznego~\cite{isp};
      \item Infrastruktura komunikacyjna
            --- Infrastruktura IoT jest bardziej elastyczna, podczas gdy infrastruktura
            IIoT wymaga dostosowania do bardziej ustrukturyzowanych modeli komunikacyjnych;
      \item Ograniczenia
            --- W otoczeniu przemysłowym charakterystyczne są wysokie wymagania dotyczące
            ograniczeń czasowych (determinizmu), niezawodności, bezpieczeństwa danych oraz
            funkcjonowania w niesprzyjających warunkach środowiskowych.
\end{itemize}

Szczególną uwagę należy poświęcić stwierdzeniu, że \textbf{intencją wykorzystania IIoT nie
      jest zastąpienie tradycyjnych systemów automatyzacji procesów przemysłowych}.
Na tej płaszczyźnie istnieją bowiem dedykowane rozwiązania adresujące
problem ograniczeń spotykanych w automatyce przemysłowej.
Właściwym celem wdrażania systemów IIoT jest zwiększanie wiedzy na temat procesów
przemysłowych, co w efekcie pozwala na ulepszenie ich wydajności \cite{iiot-challenges-opportunities-directions}.

Ze względu na tolerancję ograniczeń w systemie (m. in. opóźnień czasowych) można
wyróżnić trzy poziomy jakości usług komunikacyjnych \cite{iot-hype-to-reality}:
\begin{itemize}
      \itemsep0em
      \item Dostarczanie z wykorzystaniem najlepszych możliwości (ang. \emph{best effort})
            --- Komunikacja nie spełnia ścisłych ograniczeń czasowych, wykorzystywana jest
            maksymalna dostępna przepustowość. Jest to poziom jakości charakterystyczny
            dla usług internetowych;
      \item Dostarczanie w czasie gwarantowanym
            --- Komunikacja spełnia pewne ograniczenia czasowe, których niedotrzymanie
            skutkuje spadkiem korzyści płynących z~wykorzystawanych usług;
      \item Dostarczanie w czasie deterministycznym
            --- Komunikacja powinna spełniać ścisłe ograniczenia czasowe,
            których niedotrzymanie jest równoznaczne z awarią systemu.
\end{itemize}

Powszechna infrastruktura tworząca Internet, jak i usługi w nim dostępne, są z reguły
niedeterministyczne pod względem czasowym. Systemy IoT czy IIoT mogą więc
funkcjonować zgodnie paradygmatem \emph{best effort}. Aczkolwiek niekoniecznie
należy odrzucać możliwość współpracy systemu z układem automatyki na poziomie lokalnym,
zanim dane trafią do Internetu \cite{iiot-design-and-impl-gateway}, \cite{iiot-rapid-integration-framework}.

Dla IIoT zostały opracowane pewne wzorce architektoniczne,
które stanowią szablon dla implementacji tego typu systemów.
Obecnie popularny jest wzorzec trójwarstwowy, dla którego wyróżnia się
warstwę brzegową (ang. \emph{edge}), warstwę platformową (ang. \emph{platform})
i warstwę biznesową (ang. \emph{enterprise}).
Model ten przedstawiono na rys.~\ref{fig:iiot-arch}. Do warstwy brzegowej należą
typowe urządzenia systemów automatyki przemysłowej, m.in. czujniki i urządzenia wykonawcze oraz
sterowniki (np. klasy PLC) oraz łączące je sieci komputerowe. Warstwa platformowa umożliwia zarządzanie
systemem IoT, akwizycję i agregację danych oraz ich trasowanie na wejścia
aplikacji i usług warstwy biznesowej, które
korzystają ze zgromadzonych danych w celu wspomagania
analityki, monitorowania, zarządzania, archiwizacji, itd. \cite{iiot-challenges-opportunities-directions},  \cite{models-innovative-iot}.

\begin{figure}[h]
      \centering
      \includegraphics[scale=0.6]{iiot_arch.png}
      \caption{Trójwarstwowy wzorzec architektury systemów IIoT}
      \label{fig:iiot-arch}
\end{figure}

W wyniku połączenia idei wywodzących się z systemów IoT, IIoT i CPS
powstała koncepcja \textbf{Przemysłu 4.0} (niem. \emph{Industrie 4.0}) reprezentująca
tzw. czwartą rewolucję przemysłową. Domenę Przemysłu 4.0 opisuje na wysokim poziomie
model architektury RAMI 4.0 (ang. \emph {Reference Architectural
      Model Industrie 4.0}), któremu ustępuje
obecnie przestarzały model piramidowy opisany w rozdziale \ref{arch-przemysl}.
Architektura RAMI 4.0 została przedstawiona na rys. \ref{fig:rami}.
Jest to model trójwymiarowy opisujący trzy podstawowe aspekty Przemysłu 4.0.
Na osi o nazwie ,,Poziomy hierarchii`` umieszczono poziomy funkcjonowania
jednostek produkcyjnych. Podział ten wywodzi się ze standardów IEC 62264 i
ISA-95 (modelu piramidowego). Oś o nazwie ,,Cykl życia strumienia wartości``
reprezentuje cykl życia produktów i innych aktywów. W pierwszej fazie
następuje projektowanie, rozwój i prototypownie (powstaje tzw. ,,typ produktu``), a w
drugiej fazie rozpoczyna się produkcja (powstają tzw. ,,instancje produktu``).
Pionowa oś przedstawia sześć warstw funkcjonowania systemów informatycznych
w ramach Przemysłu 4.0. Każda z nich udostępnia usługi dla warstwy wyższej
i używa usług warstwy niższej. Warstwa aktywów (zasobów) reprezentuje fizyczne
urządzenia. Warstwa integracji jest pomostem między światem fizycznym a światem
informacji. Warstwa komunikacji zapewnia jednolity format danych i sposób komunikacji.
Warstwa informacji odpowiada za przetwarzanie danych. Warstwa funkcjonalna
dostarcza środowiska do realizacji modeli implementowanych w ramach warstwy biznesowej
\cite{iiot-challenges-opportunities-directions},
\cite{industry4-ref-models}, \cite{rami4}, \cite{iiot-cyber-manufacturing-systems}.

\begin{figure}[h]
      \centering
      \includegraphics[width=0.8\textwidth]{RAMI4.png}
      \caption{Model architektury RAMI 4.0}
      \label{fig:rami}
\end{figure}


\subsection{Ujęcie tematu w świetle zgromadzonych informacji}

Głównym zagadnieniem rozważanym
w ramach projektu jest zbieranie danych pomiarowych w środowisku
przemysłowym. Problem ten adresują technologie IIoT oraz usługi chmurowe.
Umożliwiają one gromadzenie i przetwarzanie danych w systemie rozproszonym składającym
się z szeroko pojętej aparatury kontrolno-pomiarowej (urządzeń cybernetyczno-fizycznych).
W niniejszej pracy podjęto zadanie \textbf{zaprojektowania architektury} opisanego systemu
oraz \textbf{realizacji jego prototypu}.

Sporządzenie koncepcji architektury systemu sprowadza się do wypracowania
\textbf{ogólnego rozwiązania} dla klasy zadań, jaką jest zbieranie danych w środowisku przemysłowym.
Zaproponowany model będzie więc abstrahować od konkretnych narzędzi, technologii,
urządzeń czy szczegółowych przypadków użycia. Przykładową implementację modelu ma
stanowić wykonany prototyp, na poziomie którego dobór technologii
i aparatury będzie skonkretyzowany na podstawie analizy i porównań dostępnych
narzędzi i usług informatycznych.

Modelem referencyjnym dla projektowanego systemu jest zaprezentowana na rys.~\ref{fig:iiot-arch}
\textbf{trójwarstwowa architektura systemów IIoT}. Dla warstwy brzegowej przyjmuje
się, że w systemie jest zapewniony dostęp do dwóch rodzajów urządzeń: przemysłowe urządzenia
lub zespoły urządzeń aparatury kontrolno-pomiarowej pracujące w czasie rzeczywistym
w ramach sieci przemysłowej
oraz typowe dla IoT bezprzewodowe sieci WSN/WSAN (patrz: rozdział \ref{iot})
lub inne zespoły urządzeń zawierające czujniki komunikujące się z wykorzystaniem
interfejsów i protokołów charakterystycznych dla IoT \cite{iiot-gateway-introduction}. Implementację warstwy
platformowej realizują rozwiązania oferowane przez dostawców chmury.
Do warstwy biznesowej mogą należeć istniejące aplikacje
wspomagające działalność przedsiębiorstwa
(m. in. systemy klasy ERP) oraz mogące przejmować ich funkcje usługi chmurowe.
Pewną lukę, element brakujący w proponowanym modelu stanowi konieczność dopasowania
do różnych protokołów sieci przemysłowych i IoT. Istnieje również potrzeba integracji
warstwy brzegowej i platformowej zaproponowanego modelu. Są to krytyczne zadania
przypisywane systemom IIoT. Do rozwiązania
przytoczonego problemu stosuje się tzw. \textbf{bramy IoT} (ang. \emph{IoT gateways}). Są to zwykle
urządzenia wraz z~oprogramowaniem będące w stanie integrować zarówno protokoły
warstwy fizycznej i łącza danych, jak i dokonywać translacji protokołów wyższych warstw
(tzw. bramy semantyczne) wraz z możliwością udostępniania danych za pośrednictwem
Internetu \cite{iiot-heterogenous-gateways}. Bramy mogą realizować ponadto
wstępne przetwarzanie danych odciążając w ten sposób usługi wyższych warstw.
Jest to tzw. koncepcja \textbf{przetwarzania brzegowego} (ang. \emph{edge computing}) \cite{iot-gateway-medical-and-industrial}.
W ogólniejszym rozumieniu bramy IoT mają stanowić pomost pomiędzy technologią operacyjną a
technologią informatyczną \cite{iiot-gateway-introduction}, \cite{iiot-challenges-opportunities-directions}.

Otrzymano zatem podstawowy zarys projektowanego systemu.
Składa się on z~urządzeń pomiarowych udostępniających dane w ramach sieci
przemysłowych oraz IoT. Zbieraniem danych zajmuje się brama (bramy) IoT, która
przekazuje dane w spójnym, jednolitym
formacie do usług chmurowych i innych aplikacji za pośrednictwem Internetu.
Te z kolei mogą realizować zaawansowane przetwarzanie, składowanie, udostępnianie
i wizualizację danych.

Wśród istniejących, komercyjnych rozwiązań o podobnej dziedzinie funkcjonalnej
znaleziono następujące przykłady:
\begin{itemize}
      \itemsep0em
      \item Sensemetrics --- Rozwiązanie amerykańskiej firmy Industrial IoT Solutions.
            Jest to system do akwizycji i zarządzania danymi \cite{sensmetrics};
      \item Nazca 4.0 --- System rozwijany przez polską firmę APA Group zorientowany na Przemysł 4.0.
            Wśród użytych technologii producent wymienia IoT, inteligentne czujniki,
            przetwarzanie w chmurze, integrację systemów \cite{nazca};
      \item IXON --- Rozwiązanie IIoT holenderskiej firmy o tej samej nazwie.
            System składa się z platformy internetowej umożliwiającej nadzór, analitykę i~wizualizację danych oraz
            dedykowanego routera/bramy IoT \cite{ixon}.
\end{itemize}


\newpage
\section{Specyfikacja wymagań}\label{wymagania}

W niniejszym rozdziale przedstawiono szczegółową specyfikację wymagań
stawianych wobec projektowanego systemu i prototypu. Dokumentacja składa się
z wymagań funkcjonalnych i niefunkcjonalnych, przypadków użycia oraz scenariuszy
biznesowych implementowanych w ramach prototypu.

\subsection{Wymagania funkcjonalne}

Do założonych ogólnych funkcjonalności systemu należą: zbieranie, przetwarzanie,
składowanie, udostępnianie i wizualizacja danych pomiarowych, które omówiono
w kolejnych podrozdziałach.

\subsubsection{Zbieranie danych pomiarowych}

Zbieranie (gromadzenie) danych pomiarowych to podstawowe zadanie projektowanego systemu.
Pod pojęciem ``dane pomiarowe'' rozumie się opis obiektów i zjawisk fizycznych
zapisany w postaci cyfrowej, który jest zwykle przyporządkowany do określonego
punktu lub przedziału w czasie. Ciąg takich opisów nazywa się szeregiem czasowym \cite{time-series}.
Do możliwych źródeł danych pomiarowych należą: aparatura pomiarowa (czujniki, sensory),
pamięć masowa, użytkownicy wchodzący w~interakcję z systemem oraz inne systemy informatyczne.
Projektowany system powinien \textbf{udostępniać interfejs wejściowy} umożliwiający zbieranie
i przekazywanie danych pochodzących z~wymienionych źródeł. Należy ponadto uwzględnić
fakt, że urządzenia dostarczające dane mogą komunikować się na różne sposoby.
Stąd konieczne jest, aby system był przystosowany do współpracy z popularnymi
interfejsami oraz protokołami komunikacyjnymi.

\subsubsection{Przetwarzanie danych}

Realizowany system ma za zadanie umożliwić \textbf{przetwarzanie zgromadzonych danych pomiarowych}
--- zarówno na poziomie sprzętowym, jak i~programowym. Wśród możliwych przykładów
przetwarzania danych w systemie można wymienić: agregację danych
(scalanie wyników, wyznaczanie statystyk),
obliczanie kluczowych wskaźników efektywności prowadzonej działalności
(ang. \emph{Key Performance Indicators}, w skrócie: KPI), uczenie maszynowe
(zadania klasyfikacji, prognozowanie, optymalizacja procesów), automatyzację zadań.

\subsubsection{Składowanie danych}

W ramach dziedziny biznesowej istnieje zwykle potrzeba przyszłego wykorzystania
zebranych danych oraz wyników ich przetwarzania. Trwały zapis danych dokumentuje
procesy zachodzące w przedsiębiorstwie i udogadnia sprawowanie kontroli; pozwala
na gromadzenie większej ich ilości i w konsekwencji pogłębioną, bardziej dokładną
analizę. Zatem oczekuje się, aby projektowany system \textbf{realizował
      funkcjonalność składowania (archiwizacji) danych pomiarowych w bazach danych}
i umożliwiał dostęp do nich w przyszłości.

\subsubsection{Wizualizacja danych}

Aby gromadzenie i przetwarzanie danych pomiarowych mogło przynieść jakiekolwiek
korzyści dla przedsiębiorstwa niezbędne jest, aby dane otrzymywane na wyjściu realizowanego
systemu posiadały reprezentację interpretowalną, łatwo przyswajalną dla człowieka.
Możliwe jest wówczas dokonywanie dalszej analizy i zadań obserwacji, monitorowania
czy zarządzania. System \textbf{powinien zatem udostępniać przyjazny dla człowieka
      interfejs (graficzny)}, w którym dane przedstawione są za pomocą tabel, wykresów,
diagramów, plików o przejrzystej strukturze.

\subsubsection{Udostępnianie danych}

Z efektów działania projektowanego systemu bezpośrednio mogą korzystać nie tylko
ludzie, lecz także inne systemy. Wymaga się więc, aby realizowany system \textbf{mógł udostępniać
      zgromadzone dane i wyniki ich przetwarzania na potrzeby zewnętrznych systemów (klientów)},
które realizują kolejne etapy przetwarzania danych.

\subsection{Wymagania niefunkcjonalne}

Rozdział zawiera opis następujących wymagań niefunkcjonalnych:
wykorzystanie technologii IoT i przetwarzania w chmurze, dostosowanie do środowiska
przemysłowego, ogólność rozwiązania, skalowalność i bezpieczeństwo danych. Szczegółowe
rozważania dotyczące wymienionych założeń przedstawiono w ramach kolejnych
podrozdziałów.

\subsubsection{Wykorzystanie technologii IoT i przetwarzania w chmurze}

Wykorzystanie Internetu Rzeczy to podstawowe założenie, które zostało ujęte
w~temacie niniejszej pracy i zaznaczone przy określeniu
celu projektu. Do najważniejszych zastosowań IoT należy zbieranie dużej ilości
danych z wielu różnych urządzeń. Użycie rozwiązań z tej dziedziny jest więc wyborem uzasadnionym.
Zadaniem krytycznym w systemach IoT jest przetwarzanie zebranych danych.
Problem ten podejmują przede wszystkim oferowane za pośrednictwem Internetu
usługi chmurowe. Tak więc Internet Rzeczy i~przetwarzanie w~chmurze w~stopniu
znaczącym odpowiadają na większość wymagań funkcjonalnych przedstawionych w~poprzednim rozdziale.

\subsubsection{Dostosowanie do środowiska przemysłowego}

Drugim założeniem podstawowym obejmującym zakres niniejszej pracy
jest \textbf{przystosowanie projektowanego systemu do zastosowania w przemyśle.}
W wielu przedsiębiorstwach istnieją już sprawdzone systemy automatyki wyposażone
w aparaturę pomiarową. Jako że rozwój technologiczny w przemyśle ma z reguły
charakter stopniowy, wymaga się, aby projektowany system \textbf{integrował się
      z obecnie funkcjonującymi systemami}, co oznacza konieczność posiadania
odpowiednich interfejsów komunikacyjnych i implementowania określonych protokołów
charakterystycznych dla rozproszonych systemów przemysłowych.

Jak już wspomniano w rozdziale \ref{analiza}, systemy kontrolujące
procesy przemysłowe muszą zwykle działać w ramach twardych ograniczeń czasowych.
Natomiast wykorzystanie technologii internetowych gwarantuje działanie na poziomie
\emph{best effort}. Celem ich stosowania w przemyśle nie jest jednak zastąpienie
istniejących systemów automatyki, lecz ich integracja z systemami warstwy IT,
które niekoniecznie operują w sztywnych ramach czasowych.
Aczkolwiek, żeby zapewnić szerszą współpracę projektowanego systemu
z istniejącymi rozwiązaniami przemysłowymi, należy \textbf{uwzględnić możliwość jego funkcjonowania
      w czasie rzeczywistym na poziomie lokalnym}, tzn. w ramach warstwy brzegowej
(patrz: rozdział \ref{analiza}, rys. \ref{fig:iiot-arch}).

W środowisku przemysłowym oczekuje się zwiększonej niezawodności systemów
zarówno na poziomie sprzętowym, jak i programowym. Choć w przypadku IIoT nie jest
to zwykle wymaganie o stopniu tak krytycznym, jak w standardowych systemach automatyki,
to w projekcie systemu należy uwzględnić w zakresie podstawowym
zagadnienia \textbf{redundancji, diagnostyki i możliwości funkcjonowania w niesprzyjających
      warunkach środowiskowych}.

\subsubsection{Ogólność rozwiązania}

Pożądaną cechą projektowanego systemu jest jego uniwersalność. Wypracowane rozwiązanie
powinno być możliwie niezależne od specyficznych urządzeń, narzędzi technicznych,
oprogramowania, czy dostawców chmury. Takie podejście pozwala na implementację różnych przypadków biznesowych
oraz dostosowanie do istniejącej w przedsiębiorstwach infrastruktury sprzętu i oprogramowania
z uwzględnieniem indywidualnych preferencji, wymagań i posiadanych zasobów.

Integrację różnych systemów informatycznych na odpowiednich poziomach abstrakcji
zapewniają sieci komputerowe, interfejsy i protokoły.
Stąd \textbf{kluczowym aspektem projektowanego systemu --- co już kilkukrotnie zostało podkreślone ---
      jest dostarczenie wsparcia dla popularnych
      interfejsów, protokołów i modeli wymiany informacji stosowanych w~IoT, przemyśle oraz
      aplikacjach internetowych i usługach chmurowych}.

Wraz z implementacją popularnych protokołów wymagane jest ponadto \textbf{zapewnienie
      jednolitego formatu danych na poziomie warstwy aplikacji}, który ma zagwarantować przenośność,
łatwą wymianę danych pomiędzy współpracującymi systemami. Formaty
te powinny być uprzednio specyfikowane, a projektowany system powinien być do
nich zaadaptowany --- analogicznie, jak w przypadku protokołów.

Obok przystosowania do różnych sposobów komunikacji ważnym kryterium uniwersalności
systemu jest elastyczność oprogramowania.
W systemie, w którym zbierane są dane z wielu różnych urządzeń, wymagania dotyczące
ich przetwarzania podlegają częstej zmianie i ewolucji. \textbf{Oczekuje się zatem,
      że system będzie umożliwiał łatwą modyfikację i rozbudowę realizowanej logiki, jak
      również wymianę komponentów}. Ponadto należy uwzględnić fakt, że w zarządzaniu
systemem przemysłowym niekoniecznie muszą uczestniczyć zawodowi programiści,
z czym wiąże się potrzeba, aby \textbf{model programistyczny systemu był zrozumiały
      także dla specjalistów innych dziedzin techniki i analityków}.

\subsubsection{Skalowalność}

Projekt powinien przewidywać możliwość zmiany skali infrastruktury przedsiębiorstwa,
w ramach której ma on być wdrożony. Wymaga się, aby system mógł być
w~stosunkowo prosty, szybki i tani sposób zaadaptowany do nowych warunków pracy,
które są rezultatem zmieniającej się liczby obsługiwanych
urządzeń i~ilości koniecznych do przetworzenia danych. Dostosowanie systemu sprowadza
się w głównej mierze do przydzielenia odpowiednich zasobów sprzętowych i programowych.

\subsubsection{Bezpieczeństwo danych}

Wymaganiem stawianym wobec większości systemów informatycznych jest bezpieczeństwo
danych. Również w środowisku przemysłowym przetwarzane dane są zwykle traktowane
jako poufne, zaś ograniczenie dostępu do systemów sterowania ma charakter krytyczny,
gdyż mają one realny wpływ na bezpieczeństwo zasobów przedsiębiorstwa~--- w tym także
i ludzi. W ramach projektu należy rozważyć w zakresie podstawowym kwestie
bezpieczeństwa w aspekcie autoryzacji dostępu do systemu (w świecie rzeczywistym oraz
cyfrowym), a także zapewnienia poufności z wykorzystaniem szyfrowania i
metod ochrony przed kradzieżą.

\subsection{Przypadki użycia}\label{use-case}

Ogólne ujęcie funkcjonalności projektowanego systemu z perspektywy użytkowników,
zewnętrznych urządzeń i systemów modeluje diagram przypadków użycia przedstawiony
na rys. \ref{fig:use-case}. Na jego potrzeby zdefiniowano abstrakcyjne
źródło danych, które mogą stanowić zróżnicowane urządzenia pomiarowe, komputery,
jak również inne systemy. Źródło przesyła
dane do systemu, który dokonuje ich konwersji na pożądany format i umożliwia
dalsze składowanie, przetwarzanie, udostępnianie i wizualizację. Dane są
udostępniane na potrzeby innych systemów informatycznych (aplikacji klienckich).
Są one także bezpośrednio prezentowane pracownikom obsługi procesu przemysłowego
i analitykom w ramach wizualizacji. Wymienieni pracownicy we
współpracy z programistami i testerami uczestniczą
w procesie implementacji, modyfikacji i rozbudowy logiki przetwarzania danych.
Administrator systemu zajmuje się jego konfiguracją i utrzymaniem
poszczególnych jego komponentów. Praca ta polega na
instalacji urządzeń warstwy sprzętowej, zarządzaniu
użytkownikami i dostępem do systemu, integracji poszczególnych jego komponentów,
ustawieniu parametrów połączeń sieciowych i protokołów, monitorowaniu
i usuwaniu awarii.

\begin{figure}
      \centering
      \includegraphics[width=\textwidth]{use-case.png}
      \caption{Diagram przypadków użycia systemu.}
      \label{fig:use-case}
\end{figure}

\subsection{Scenariusze wykorzystania systemu}\label{scenariusze}

Na potrzeby walidacji funkcjonalności realizowanego projektu dokonano wyboru przykładowych
scenariuszy biznesowych, które mają zostać zaimplementowane przez prototyp.
Przyjęto założenie, że w środowisku docelowym w obrębie lokalnej sieci
są zainstalowane urządzenia pomiarowe, które komunikują się odpowiednio
z wykorzystaniem protokołów
MQTT, CoAP oraz Modbus/TCP. Pierwsze dwa są charakterystyczne dla IoT, natomiast
ostatni jest popularny wśród rozwiązań przemysłowych. System ma za zadanie
integrować się z lokalną siecią i wymienionymi protokołami, z użyciem
których gromadzone będą dane. Szczegółowy opis przyjętych scenariuszy biznesowych
przedstawiono w kolejnych podrozdziałach.

\subsubsection{Monitorowanie parametrów powietrza}

W określonych pomieszczeniach zakładu produkcyjnego zainstalowano sieć bezprzewodowych
czujników temperatury i wilgoci WiFi, która jest połączona z siecią lokalną.
Czujniki potrafią komunikować się z wykorzystaniem protokołu MQTT/TCP. Do monitorowanych
pomieszczeń należą: chłodnia, magazyn i hala produkcyjna. Prototyp systemu ma za zadanie
zbierać, składować, udostępniać i wizualizować zmierzone parametry powietrza.

\subsubsection{Monitorowanie stanu zużycia chemii przemysłowej}

W zakładzie produkcyjnym rozlokowane są trzy zbiorniki (bufory) chemii przemysłowej.
Zbiorniki są połączone z zewnętrznym zbiornikiem głównym, z którego uzupełniany
jest ich stan. W celu automatycznego napełniania zbiorników wykorzystano sterownik
klasy PLC podłączony do lokalnej sieci Ethernet, który otwiera albo zamyka
odpowiednie zawory i nadzoruje pracę pomp.
Odczytuje on dane z przepływomierzy pozwalające na monitorowanie stanu
zużycia chemii w każdej z trzech stref oraz poziom cieczy w zbiorniku głównym.
Zadaniem prototypu jest cykliczny odczyt wskazań dla każdego
zbiornika znajdujących się w pamięci PLC z wykorzystaniem protokołu Modbus/TCP.
Zebrane dane powinny być składowane w chmurze oraz wizualizowane.

\subsubsection{Wyznaczanie parametru KPI}

Na wyjściu trzech linii procesowych rozmieszczona jest aparatura przeznaczona
do oceny jakości dostarczanych produktów. Urządzenie,
w którym zapisywany jest wynik, jest podłączone do lokalnej sieci Ethernet.
Informacja o spełnieniu albo niespełnieniu wymagań jakościowych jest przesyłana
do serwera z~wykorzystaniem CoAP/UDP w oparciu o architekturę REST. Zadaniem
realizowanym przez prototyp jest odbiór przesyłanych rezultatów inspekcji,
przetwarzanie i składowanie w chmurze. Przetwarzanie
danych polega w tym przypadku na wyznaczaniu wartości parametru FTQ
(ang. \emph{First Time Quality}). Jest to jeden ze wskaźników KPI będący miarą
zdolności linii do produkowania bez wad. FTQ to iloraz liczby produktów,
które spełniły wymagania jakościowe i liczby wszystkich dostarczonych produktów
wyrażony wzorem \eqref{eq:ftq}.
Jego wartość podaje się zwykle w procentach \cite{isp}. Wskaźnik FTQ
wyznaczany dla każdej z trzech rozpatrywanych linii procesowych powinien być na bieżąco wizualizowany.

\begin{equation}
      FTQ = \frac{s}{n}\label{eq:ftq}
\end{equation}
\noindent gdzie: \\
$s$ --- liczba produktów wyprodukowanych bez wad,\\
$n$ --- liczba wszystkich produktów wyprodukowanych w procesie.


\newpage
\section{Specyfikacja wewnętrzna}\label{spec-wew}

Niniejszy rozdział zawiera specyfikację ogólnego projektu systemu oraz
opis szczegółów implementacji wykonanego prototypu.

\subsection{Ogólna koncepcja rozwiązania}\label{ogolna-koncepcja}

Proponowane rozwiązanie opiera się na trójwarstwowym wzorcu architektury
systemów IIoT (patrz: rozdział \ref{iiot}, rys. \ref{fig:iiot-arch}), który
zaadaptowano w celu praktycznego zastosowania, uwzględniwszy zdefiniowane
wymagania projektowe oraz dostępne narzędzia i usługi. Wypracowaną koncepcję
zobrazowano na rys. \ref{fig:general-idea}.

\begin{figure}[h]
      \centering
      \includegraphics[width=0.95\textwidth]{koncepcja.png}
      \caption{Ogólna koncepcja architektury systemu}
      \label{fig:general-idea}
\end{figure}

Pierwszym punktem zbierania danych w systemie
są bramy IoT (IIoT). Ich wykorzystanie
pozwala na integrację zróżnicowanej pod względem technicznym aparatury pomiarowej.
Bramy mają dostarczać odpowiednie interfejsy fizyczne, jak również dokonywać
translacji protokołów pośrednicząc w ten sposób w wymianie danych
pomiędzy warstwą brzegową i biznesową \cite{iiot-gateway-introduction},
\cite{iiot-heterogenous-gateways}, \cite{iot-gateway-medical-and-industrial},
\cite{modbus-iot-gateway}, \cite{iiot-design-and-impl-gateway}. Realizują one także
wstępne przetwarzanie w warstwie brzegowej i ułatwiają
sprawowanie kontroli nad przepływem danych w systemie \cite{iiot-architecture-and-gateway}.
Idea stosowania bramy IoT w przemyśle umożliwia spełnienie głównego zadania
IIoT, jakim jest połączenie technologii operacyjnych i informacyjnych
\cite{iiot-opensource-gateway}.
Na omawianym rysunku zamieszczono trzy bramy, które integrują urządzenia odpowiednio
w ramach sieci bezprzewodowej, przemysłowej sieci Ethernet oraz interfejsów szeregowych.
Jest to podział poglądowy. W zależności od indywidualnych potrzeb i dostępnego sprzętu możliwe są różne
konfiguracje: pojedyncza brama pracująca w wielu sieciach, bramy dedykowane
dla określonych sieci, hierarchiczna struktura bram.

Za pośrednictwem Internetu brama
przekazuje dane do warstw wyższych, z wykorzystaniem których odbywa się kolejny etap gromadzenia,
a także przetwarzanie, składowanie i udostępnianie danych. Są to warstwy platformowa
i biznesowa, których techniczną implementację umożliwiają odpowiednie usługi chmurowe
popularnych dostawców: Microsoft, Amazon, Google, Intel, IBM. Możliwe jest też
wdrożenie przez przedsiębiorstwo chmury prywatnej lub dedykowanej infrastruktury
i oprogramowania. Do warstwy biznesowej należą także aplikacje klienckie, którym są
udostępniane dane składowane w bazie. Wśród potencjalnych klientów należy wyszczególnić
oprogramowanie, które realizuje zdefiniowane w ramach projektu wymaganie wizualizacji~danych.

\subsection{Komponentowy model architektury}\label{model-komp}

Wychodząc od koncepcji opisanej w poprzednim rozdziale opracowano
formalny model architektury projektowanego systemu przedstawiony
w postaci diagramu na rys. \ref{fig:arch}. Składa się on z licznych luźno
powiązanych komponentów rozlokowanych na określonych platformach sprzętu i oprogramowania.
Rozdział systemu na pojedyncze elementy składowe odpowiada na wymaganie uniwersalności, zapewnia
lepszą konfigurowalność i skalowalność. Modularny charakter systemu
udogadnia także niezależną modyfikację i rozbudowę poszczególnych podzespołów.
Biorąc pod uwagę fakt, że system obejmuje szeroką dziedzinę (od warstwy procesowej
po wysokopoziomowe aplikacje IT) i może być utrzymywany przez zespoły o różnych
specjalnościach, jest to aspekt szczególnie ważny. Środkiem współpracy
grup komponentów osadzonych w \texttt{Bramie IIoT}, \texttt{Chmurze} i na
\texttt{Urządzeniach Klienckich} jest
Internet i popularne protokoły internetowe: HTTP, FTP, MQTT, AMQP, WebSocket, TLS.

\begin{figure}[h]
      \centering
      \includegraphics[width=\textwidth]{arch.png}
      \caption{Komponentowy model architektury systemu}
      \label{fig:arch}
\end{figure}

\subsubsection{Brama IIoT}\label{brama-iot-general}

Zamieszczona na diagramie z rys. \ref{fig:arch} \texttt{Brama IIoT} to urządzenie (system komputerowy)
posiadające odpowiednie porty (interfejsy fizyczne, porty sieciowe, porty szeregowe),
przez które może komunikować się z aparaturą pomiarową reprezentowaną przez
\texttt{Źródło danych}. Zbieranie danych w jednym miejscu, translację protokołów,
przetwarzanie brzegowe i komunikację z komponentami udostępnianymi za pośrednictwem
Internetu realizuje dedykowane \texttt{Oprogramowanie Bramy}.
Jest to komponent złożony składający z \texttt{Komponentów Przetwarzania Brzegowego}
realizujących określone funkcje przetwarzające oraz \texttt{Buforów Brzegowych},
które umożliwiają agregację i zachowanie jednolitego formatu danych wyjściowych.
Szczegóły komponentu \texttt{Oprogramowania Bramy}
przedstawiono na rys. \ref{fig:gateway_soft}.

W ramach bramy może funkcjonować również sprzętowo-programowy moduł
czasu rzeczywistego RT (ang. \emph{Real-Time}), którego celem jest współpraca
z lokalnym systemem czasu rzeczywistego.

W systemie przewiduje się istnienie wielu źródeł
danych, jak i możliwe jest wykorzystanie wielu bram. Ponadto żeby zapewnić
większą niezawodność i skalowalność,
można rozważyć budowę klastra bram z redundancją i równoważeniem obciążenia.
Jako sprzętową realizację bramy w literaturze spotyka się komputer jednopłytkowy
RaspberryPi, a także bramki z serii Siemens Simatic IOT2000
\cite{iiot-opensource-gateway}, \cite{design-impl-node-gateway},
\cite{low-cost-esp32-pi-node-red-scada}, \cite{modbus-iot-gateway}. Można też
wziąć pod uwagę wykorzystanie platformy BeagleBone Black, czy dedykowanych rozwiązań
producentów Cisco, Dell, Huawei, Hewlett Packard, NXP i innych \cite{gateways}.
Do realizacji oprogramowania można posłużyć się narzędziami przeznaczonymi do
integracji urządzeń IoT. Jednym z najpopularniejszych programów jest
darmowe narzędzie Node-RED \cite{flow-programming}, \cite{iot-gateway-medical-and-industrial},
\cite{design-impl-node-gateway}, \cite{iiot-opensource-gateway}, \cite{low-cost-esp32-pi-node-red-scada}.
Alternatywne bezpłatne rozwiązania to m.in. n8n.io, Total.js Flow.
Wśród ofert komercyjnych dostępne są przykładowo platformy Crosser, Bosch ProSyst
\cite{node-red}, \cite{n8n}, \cite{total-js-flow}, \cite{crosser}, \cite{gateways}.

\begin{figure}
      \centering
      \includegraphics[scale=0.4]{oprog_bramy.png}
      \caption{Model oprogramowania bramy IIoT}
      \label{fig:gateway_soft}
\end{figure}

\subsubsection{Chmura}\label{cloud-general}
Przedstawiony na rys. \ref{fig:arch} diagram o nazwie \texttt{Chmura}
może w rzeczywistości odpowiadać chmurze publicznej, prywatnej lub innej
dedykowanej infrastrukturze sprzętu i oprogramowania. W niniejszym projekcie
w szczególności rozpatruje się wariant z chmurą publiczną, której usługi są dostępne
za pośrednictwem Internetu. Pozwalają one na techniczną realizację postawionych
wymagań przetwarzania, składowania i udostępniania danych.
Umieszczone wewnątrz diagramu komponenty zostały dobrane
pod kątem zdefiniowanych wymagań funkcjonalnych. \texttt{Platforma IoT}
to komponent, który odbiera dane przychodzące z urządzeń (bram IIoT).
Umożliwia również zarządzanie nimi:
dodawanie, usuwanie, modyfikację uprawnień, a także monitoring i diagnostykę.
Dane z platformy mogą zostać bezpośrednio składowane w \texttt{Bazie Danych}.
Możliwe jest też przekazanie ich na wejście \texttt{Komponentu Przetwarzającego}.
Pod tym pojęciem rozumiane są wszelkie usługi chmurowe realizujące tzw. bezserwerowe
przetwarzanie danych (ang. \emph{serverless computing}). Nie oznacza to, że
serwer nie istnieje, lecz że jest on niejako ,,zakryty'' przed programistą, który skupia
się jedynie na dostarczeniu logiki w postaci wsadowej \cite{azure-serverless}.
Usługi przetwarzania bezserwerowego dokonują ewentualnego odczytu i zapisu do \texttt{Bazy Danych}.
Wśród usług oferowanych przez liderów rynku chmurowego
--- Amazon Web Services (AWS) i Microsoft Azure \cite{gartner-cloud-liders}
--- dokonano przeglądu przykładowych, które mogą być użyte do implementacji opisanych komponentów.
W poniższych punktach oddzielono znakiem ,, | '' odpowiednio nazwy usług AWS i Azure,
które mają podobny zakres funkcjonalny \cite{azure-aws-comparison}.\\

\noindent\textbf{Przykładowe usługi użyteczne do realizacji \texttt{Platformy IoT}}:
\begin{itemize}
      \itemsep0em
      \item IoT | IoT Hub ---  Brama w chmurze umożliwiająca komunikację z
            urządzeniami w bezpieczny i skalowalny sposób,
      \item IoT Things Graph | Digital Twins --- Tworzenie cyfrowych modeli systemów
            fizycznych umożliwiających zbieranie danych ze świata rzeczywistego.
\end{itemize}

\noindent\textbf{Przykładowe usługi użyteczne do realizacji \texttt{Bazy Danych}}:
\begin{itemize}
      \itemsep0em
      \item Simple Storage Services | Blob Storage --- Magazyn do składowania,
            udostępniania, archiwizowania i tworzenia kopii zapasowej obiektów,
      \item RDS | SQL Database --- Relacyjne bazy danych,
      \item DynamoDB | CosmosDB --- Nierelacyjne bazy danych,
\end{itemize}

\noindent\textbf{Przykładowe usługi użyteczne do realizacji \texttt{Komponentów Przetwarzania}}:
\begin{itemize}
      \itemsep0em
      \item Lambda | Functions --- Bezserwerowe, skalowalne funkcje implementowane,
            z~wykorzystaniem języków programowania: Javascript, Python, Java, C\#,
      \item SageMaker | Machine Learning --- Uczenie maszynowe,
      \item Kinesis Analytics | Stream Analytics --- Tworzenie potoków przetwarzania
            strumieniowego dla dużej ilości danych z wykorzystaniem języka SQL.
\end{itemize}

\subsubsection{Aplikacja kliencka}

Przedstawione na rys. \ref{fig:arch} \texttt{Urządzenie Klienta} może w rzeczywistości
być komputerem osobistym, urządzeniem mobilnym, serwerem lub usługą chmurową.
Na urządzeniu jest wdrożone \texttt{Oprogramowanie Klienta}. Jest to komponent, który
korzysta za pośrednictwem Internetu z danych udostępnianych przez \texttt{Chmurę}.
W kontekście niniejszej pracy rozpatruje się w szczególności
aplikacje wizualizujące dane i wspomagające analitykę. Wśród potencjalnie
użytecznych narzędzi dostępne są darmowe rozwiązania \emph{open source}, które
pozwalają na tworzenie modularnych paneli wizualizacji w ramach
aplikacji internetowej (przeglądarkowej): Grafana,
Freeboard, Kibana, Dash. Na rynku oferowane są także komercyjne platformy analityczne:
PowerBI, Tableau, Grow.

\subsection{Specyfikacja wewnętrzna prototypu}

Na podstawie architektury systemu opisanej w poprzednim rozdziale wykonano
prototyp, który implementuje scenariusze użycia zdefiniowane w rozdziale \ref{scenariusze}.
Model zrealizowanego prototypu przedstawiono na rys. \ref{fig:impl}.
System składa się z dwóch bram IIoT: Siemens Simatic IOT2020 i RaspberryPi.
Źródła danych
zostały zasymulowane przez komputery, na których uruchomiono specjalnie wykonane oprogramowanie
generujące zadane przebiegi. Sporządzone narzędzie do symulacji danych opisano szerzej w rozdziale
\ref{testy}. Na potrzeby translacji protokołów
i przetwarzania brzegowego w systemach obu bram zainstalowano platformę Node-RED.
W celu dalszego zbierania, przetwarzania, składowania i udostępniania danych
użyto usług chmury Azure. Na potrzeby bieżącej wizualizacji danych wykorzystano
bazę danych szeregów czasowych InfluxDB osadzoną w chmurze InfluxDB Cloud.
Zebrane tam dane pomiarowe są cyklicznie odczytywane i wizualizowane przez narzędzie Grafana.
W kolejnych rozdziałach omówiono poszczególne elementy systemu.

\begin{figure}[h]
      \centering
      \includegraphics[width=\textwidth]{impl.png}
      \caption{Model implementacji prototypu}
      \label{fig:impl}
\end{figure}

\subsubsection{Bramy IIoT --- Simatic IOT2020 i RaspberryPi}

Wyboru urządzenia IOT2020 dokonano ze względu na spełnienie wymaganych standardów przemysłowych,
o czym zapewnia producent. Posiada ono certyfikat UL/CE. Płytka elektroniczna
jest umieszczona w obudowie, która jest przystosowana do montażu
na popularnych w przemyśle szynach DIN i zapewnia ochronę o stopniu IP20. Bramka
ma wbudowane interfejsy USB, Ethernet oraz moduł rozszerzeń GPIO. W systemie
komputerowym Intel Quark x1000 400 MHz, 512 MB RAM można uruchomić system
Linux z obrazu na karcie microSD. Producent dostarcza obraz systemu zbudowany
na dystrybucji Yocto Linux dedykowanej dla urządzeń wbudowanych. Udostępniona
jest także wersja z modułem czasu rzeczywistego i sterownikami Profinet,
dzięki której brama IIoT może funkcjonować z lokalnym systemem czasu rzeczywistego
\cite{simatic-iot-sepc}, \cite{simatic-iot-profinet}, \cite{modbus-iot-gateway}.

Jako że brama IOT2020 zapewnia lepsze przystosowanie do środowiska przemysłowego,
postanowiono wykorzystać ją do zbierania danych ze sterownika PLC, który
monitoruje stan zużycia chemii przemysłowej. Komunikacja odbywa się w ramach
sieci Ethernet z wykorzystaniem protokołu Modbus/TCP. Modbus to protokół,
w którym występuje węzeł nadrzędny (ang. \emph{Master}) oraz węzły
podrzędne (ang. \emph{Slave}). Węzeł nadrzędny może dokonywać odczytu i zapisu
do węzła podrzędnego z wykorzystaniem odpowiednich funkcji kodowanych w ramce.
Pierwotnie protokół był zaprojektowany dla połączeń szeregowych, z czasem
zaczęto używać go także na stosie TCP/IP kapsułkując ramki Modbus w ramce TCP
\cite{isp}, \cite{isp-analiza-przepływu-informacji}. W rozpatrywanej konfiguracji
brama pełni rolę węzła nadrzędnego, który cyklicznie odczytuje zawartość
rejestrów węzła podrzędnego (sterownika PLC).

Minikomputer RaspberryPi nie spełnia ściśle wymagań przemysłowych, natomiast ze
względu na dostępne wsparcie użyto go na potrzeby prototypownia i rozwoju.
Jest to bowiem platforma bardzo popularna, posiadająca liczną społeczność zwolenników.
RaspberryPi jako bramę IoT/IIoT można spotkać również w literaturze \cite{iiot-opensource-gateway},
\cite{design-impl-node-gateway}, \cite{low-cost-esp32-pi-node-red-scada}.
W projekcie użyto modelu 3 B, który posiada m. in. czterordzeniowy
procesor o taktowaniu 1.2 GHz, 1 GB RAM, interfejsy USB, Ethernet, Bluetooth, GPIO.
Z karty microSD można uruchomić dedykowaną dystrybucję systemu Linux RaspberryPi OS lub inną.

Bramę RaspberryPi wykorzystano do komunikacji z siecią bezprzewodowych czujników
mierzących parametry powietrza w określonych pomieszczeniach oraz z
komputerami zainstalowanymi przy trzech stacjach procesowych, które
przesyłają informacje o ocenie jakości wytworzonych produktów.
Komunikacja z czujnikami odbywa się z wykorzystaniem sieci WiFi i protokołu
MQTT. MQTT (ang. \emph{Message Queue Telemetry Transport}) to lekki pod względem wykorzystania zasobów
protokół polegający na wymianie komunikatów. Komunikacja
jest oparta na wzorcu publikuj-subskrybuj (ang.~\emph{publish-subscribe}).
Klienci łączą się do centralnego serwera (pośrednika, ang. \emph{broker}), za pośrednictwem
którego mogą wysyłać i odbierać komunikaty pod konkretnym adresem (tematem, ang. \emph{topic}).
Tematy tworzą hierarchiczną strukturę analogiczną do URL. MQTT
wykorzystuje TCP w warstwie transportowej \cite{iot-hype-to-reality}. W omawianym
scenariuszu czujniki publikują dane pomiarowe pod tematami zarezerwowanymi
odpowiednio dla pomieszczenia chłodni, magazynu i hali produkcyjnej. Brama
jest subskrybentem, który nasłuchuje na każdym z trzech tematów.

Wymiana danych z komputerami, które zbierają informacje na temat jakości produktów odbywa się
za pośrednictwem sieci Ethernet i protokołu CoAP. CoAP (ang. \emph{Constrained Application Protocol})
to alternatywa dla HTTP przeznaczona dla urządzeń o ograniczonych zasobach.
Wykorzystuje m. in. protokół UDP w warstwie transportowej oraz krótsze nagłówki w formie binarnej.
CoAP jest zorientowany na zasoby umożliwiając ich tworzenie, odczyt, aktualizację
i usuwanie. Analogicznie do HTTP realizowany jest model komunikacji
klient-serwer z użyciem metod GET, POST, PUT, DELETE, itd. \cite{intro-to-iot}, \cite{iot-hype-to-reality}.
W realizowanym przypadku biznesowym brama jest serwerem, a trzy komputery
przy stacji procesowej klientami, które przesyłają dane metodą POST.

\subsubsection{Oprogramowanie bram --- Node-RED}
Komponent \texttt{Oprogramowania} obu bram (patrz: rys. \ref{fig:gateway_soft} , rozdział \ref{model-komp})
zaimplementowano z wykorzystaniem
narzędzia Node-RED. Jest to darmowe oprogramowanie typu \emph{open source} funkcjonujące
w środowisku Node.js. Jego celem jest integracja interfejsów
fizycznych oraz interfejsów programowania aplikacji.
W Node-RED stosuje się paradygmat programowania opartego na przepływie danych.
Program posiada postać wizualną, w ramach której łączy się węzły (ang. \emph{nodes})
realizujące określone funkcjonalności. Pomiędzy wyjściem i wejściem kolejnych
węzłów przekazywany jest komunikat (wiadomość, ang. \emph{message}) w powszechnym formacie
JSON (ang. \emph{JavaScript Object Notation}). Węzły mogą być prostymi lub złożonymi
funkcjami, a nawet rozbudowanymi bibliotekami napisanymi w języku JavaScript.
Wizualny model programowania odpowiada na wymaganie uniwersalności,
jako że może być on łatwo zrozumiały dla specjalistów różnych dziedzin pracujących w przemyśle
\cite{flow-programming}.
Node-RED to serwer, którego panel administracyjny i środowisko programistyczne
udostępniane są w ramach aplikacji przeglądarkowej. Wyboru opisanego
narzędzia dokonano ze względu na elastyczność sposobu implementacji logiki i
wsparcie w postaci wielu bibliotek umożliwiających translację protokołów.
Dystrybucja w formie \emph{open source} daje większą kontrolę nad używanym oprogramowaniem,
zaś licencja Apache 2.0 pozwala na jego bezpłatne wykorzystanie również
w celach komercyjnych \cite{node-red}. Node-RED jest stosowane w IoT stosunkowo często \cite{iot-gateway-medical-and-industrial},
\cite{design-impl-node-gateway}, \cite{iiot-opensource-gateway}, \cite{low-cost-esp32-pi-node-red-scada}.

Zadaniem bramy zdefiniowanym w obrębie wymagań jest zapewnienie jednolitych
struktur danych stosowanych w warstwie aplikacji na poziomie wszystkich
systemów informatycznych w przedsiębiorstwie. W tym celu proponuje się
wykorzystanie formatu JSON. Za pomocą notacji JSON Schema możliwe jest
formalne określanie struktur danych, które powinny być zaimplementowane przez
wszystkie systemy.

Struktury danych niezbędne do realizacji przypadków biznesowych
przedstawiono na rys. \ref{fig:data-structs}. Zdefiniowano następujące klasy:
\begin{itemize}
      \itemsep0em
      \item \texttt{Measurement} --- Klasa abstrakcyjna reprezentująca pomiar, który ma określony typ, np. pomiar powietrza.
            Jawna informacja o typie może być zbędna na poziomie języka programowania, lecz musi być zawarta w obiekcie JSON, który podlega serializacji;
      \item \texttt{FluidControlMeasurement} --- Reprezentuje informację o poziomie zużycia cieczy w strefach 1, 2, 3 oraz o poziomie cieczy w zbiorniku głównym;
      \item \texttt{AirMeasurement} --- Reprezentuje pomiar powietrza. Posiada kolejno pola informujące o lokacji, wartości temperatury i wilgotności;
      \item \texttt{ProductEvaluationMeasurement} --- Reprezentuje ocenę jakości produktu. Posiada kolejno pola informujące o nazwie produktu,
            spełnieniu wymagań (tak/nie --- \texttt{true}/\texttt{false}) i ewentualnej przyczynie niespełnienia wymagań.
\end{itemize}
\noindent Przykładową definicję struktury \texttt{AirMeasurement} w notacji
JSON Schema przedstawiono na listingu \ref{lst:json-schema}.

\begin{figure}[h]
      \centering
      \includegraphics[scale=0.6]{data_structs.png}
      \caption{Struktury danych używane w warstwie biznesowej systemu}
      \label{fig:data-structs}
\end{figure}

\begin{lstlisting}[caption={Przykład definicji obiektu JSON w notacji JSON Schema dla 
      struktury \texttt{AirMeasurement}}, label={lst:json-schema}]
{
      "anyOf": [
              {
                "type": "object",
                "required": [
                  "type",
                  "locationName",
                  "temperature",
                  "humidity"
                ],
                "properties": {
                  "type": {
                    "type": "string"
                  },
                  "locationName": {
                    "type": "string"
                  },
                  "temperature": {
                    "type": "number"
                  },
                  "humidity": {
                    "type": "number"
                  }
                }
              }
      ]
}    
\end{lstlisting}

\subsubsection{Bufor Brzegowy w Node-RED}
Na potrzeby realizacji komponentu \texttt{Bufora Brzegowego} (patrz: rozdział \ref{brama-iot-general})
zaprojektowano własny węzeł Node-RED o nazwie \texttt{EdgeBuffer}.
Jego model przedstawiono na rys. \ref{fig:edge-buffer}.
Bufor ma być w domyśle umieszczany tuż przed węzłem wysyłającym dane na
zewnątrz systemu. Jego rolą jest sprawdzenie, czy format
przekazanych danych jest zgodny ze strukturami zdefiniowanymi za pomocą notacji JSON Schema.
Ma on także umożliwić
kumulowanie danych w celu ich agregacji. Bufor posiada następujące
metody publiczne:
\begin{itemize}
      \itemsep0em
      \item \texttt{send} --- przepisanie danych z wejścia na wyjście z
            opcjonalnym opóźnieniem i obowiązkową uprzednią walidacją,
      \item \texttt{put} --- składowanie danych wejściowych w buforze po uprzedniej walidacji,
      \item \texttt{flush} --- przekazanie składowanych danych na wyjście
            i zwolnienie bufora.
\end{itemize}
\noindent W ramach implementacji \texttt{BufferImpl} zdefiniowano
prywatną kolejkę \texttt{queue}, która umożliwia przekazanie zbuforowanych
danych na wyjście w kolejności FIFO. Prywatna metoda \texttt{validate} realizuje
logikę walidacji formatu danych wejściowych i zwraca wartość \texttt{true},
gdy dane są poprawne. W konfiguracji Node-RED przewidziano parametr określający
ścieżkę do pliku zawierającego definicje prawidłowych struktur danych w notacji JSON Schema.

\begin{figure}
      \centering
      \includegraphics[scale=0.4]{edge_buffer.png}
      \caption{Model \texttt{Bufora Brzegowego}}
      \label{fig:edge-buffer}
\end{figure}

\subsubsection{Zbieranie informacji o stanie zużycia chemii przemysłowej }

Diagram przepływu Node-RED realizujący logikę zbierania danych ze sterownika
PLC przedstawiono na rys. \ref{fig:flow3}. Implementację protokołu
Modbus/TCP dostarcza biblioteka \texttt{node-red-contrib-modbus}.
Jednym z jej komponentów jest węzeł \texttt{Modbus Read}, który implementuje
węzeł nadrzędny (w sensie protokołu Modbus) umożliwiający cykliczny odczyt
danych ze sterownika PLC będącego węzłem podrzędnym. Przyjęto, że dane
o zużyciu cieczy przemysłowej w strefach 1, 2, 3 są zapisane w kolejnych rejestrach
40001, 40002, 40003 jako liczby całkowite określające zużycie w litrach.
Podobnie informacja o poziomie cieczy w zbiorniku głównym zapisana jest w rejestrze 40004.
Węzeł \texttt{Modbus Read} skonfigurowano do użycia funkcji Modbus
\texttt{FC 3: Read Holding Registers} w celu odczytu danych z czterech sąsiednich
rejestrów, począwszy od adresu 40001. Odczyt wykonywany jest co 3 sekundy.
Funkcja \texttt{adapt and send} przekształca otrzymaną na wejściu tablicę
wartości na oczekiwaną strukturę \texttt{FluidControlMeasurement}
i przekazuje dane do bufora wywołując metodę \texttt{send}.
Węzeł \texttt{fluids-azure} to \texttt{Bufor Brzegowy}, którego rola sprawdza się
w rozpatrywanym przypadku jedynie do sprawdzenia poprawności otrzymanej na
wejściu struktury i natychmiastowym przekazaniu jej na wyjście. Na końcu
umieszczono węzeł \texttt{IoT Hub} pochodzący z biblioteki \texttt{node-red-contrib-azure-iot-hub},
który implementuje interfejs usługi IoT Hub chmury Azure. Umożliwia on przekazywanie
danych do chmury za pośrednictwem Internetu.
Węzeł \texttt{debug} wyświetla w konsoli informacje o ewentalnych błędach, w
tym m. in. komunikaty bufora o niepowodzeniu walidacji.

\begin{figure}[h]
      \centering
      \includegraphics[width=0.75\textwidth]{flow3.png}
      \caption{Diagram przepływu Node-RED realizujący odczyt danych ze sterownika PLC}
      \label{fig:flow3}
\end{figure}

\subsubsection{Zbieranie pomiarów parametrów powietrza }

Na rys. \ref{fig:flow1} zamieszczono diagram przepływu wykonujący gromadzenie pomiarów parametrów
powietrza w ramach protokołu MQTT. Serwer pośredniczący w wymianie komunikatów
(broker) zralizowano z wykorzystaniem darmowego oprogramowania
Eclipse Mosquitto zainstalowanego na RaspberryPi \cite{mosquitto}. Węzeł
\texttt{/sensors/freezer} to subskrybent MQTT dostępny w ramach standardowej
biblioteki Node-RED nasłuchujący na temacie o tej samej nazwie. Publikacji
komunikatów  dokonują w tym przypadku czujniki rozmieszczone w
chłodni. Identyczne diagramy, lecz z dopasowanymi tematami \texttt{/sensors/warehouse} i
\texttt{/sensors/production} opracowano dla czujników osadzonych odpowiednio
w magazynie i hali produkcyjnej. Węzeł \texttt{JSON Parser} deserializuje
nadesłane wiadomości z postaci znakowej na obiekt JSON. Funkcja \texttt{adapt and put}
mapuje dane otrzymane z czujników na strukturę \texttt{AirMeasurement}
i składa w buforze \texttt{freezer-azure}, który standardowo waliduje format
danych na wejściu. Węzeł \texttt{flush 60 sec} co minutę wywołuje metodę \texttt{flush}
zwalniającą bufor. Funkcja agregująca \texttt{avg} wylicza średnią z zebranych
wartości i wysła pojedynczą wiadomość do serwisu IoT Hub. Funkcja \texttt{adapt and send}
dostosowuje format danych do bazy InfluxDB, która wymaga przekazania
prostego obiektu JSON składającego się jedynie z pól temperatury i wilgotności.
Tylko te dane telemetryczne są wykorzystywane do wizualizacji. Na buforze
\texttt{freezer-influx} wywoływana jest metoda \texttt{send}. Dalej dane
przekazywane są do węzła \texttt{influx /freezer} pochodzącego z biblioteki
\texttt{node-red-contrib-influxdb}. Przekazuje on otrzymane dane pomiarowe
do chmury InfluxDB Cloud dołączając do nich atrybut \texttt{measurement}
określający rodzaj pomiaru --- w tym przepadku jego wartość to \texttt{"freezer"}.
Dla pozostałych monitorowanych pomieszczeń używane są wartości \texttt{"warehouse"}
oraz \texttt{"production"}.

\begin{figure}[h]
      \centering
      \includegraphics[width=\textwidth]{flow1.png}
      \caption{Diagram przepływu Node-RED realizujący zbieranie danych dostarczanych z czujników mierzących parametry powietrza}
      \label{fig:flow1}
\end{figure}

\subsubsection{Zbieranie danych do wyznaczenia parametru KPI }

Algorytm przetwarzania danych zbieranych w ramach protokołu CoAP z trzech stacji
kontroli jakości przedstawiono na rys. \ref{fig:flow2}. Węzeł \texttt{[POST] /ftq}
pochodzi z biblioteki \texttt{node-red-contrib-coap} i implementuje serwer CoAP,
a ściślej --- obsługę punktu końcowego \texttt{/ftq}, na który klienty zainstalowane
przy stacjach przesyłają wyniki ewaluacji. Funkcja \texttt{CoAP OK}
wysyła pozytywną odpowiedź na zapytanie i przekazuje otrzymane dane do kolejnych węzłów.
Dalszy przebieg algorytmu jest zbliżony do wcześniej opisanych przypadków,
gdzie dane są walidowane przez bufor i wysyłane do~serwisu IoT~Hub.

\begin{figure}[h]
      \centering
      \includegraphics[width=\textwidth]{flow2.png}
      \caption{Diagram przepływu Node-RED realizujący zbieranie danych dostarczanych ze stacji kontroli jakości}
      \label{fig:flow2}
\end{figure}


\subsubsection{Użycie chmury Azure}

Chmura Microsoft Azure udostępnia usługi, które pozwalają na realizację wszystkich komponentów zdefiniowanych
na potrzeby projektu (patrz: rozdział \ref{cloud-general}).
Z kolei Node-RED posiada dobre wsparcie dla Azure w postaci licznych bibiliotek.
Ponadto Microsoft oferuje studentom darmową subskrypcję \emph{Azure for Students}
dającą bezpłatny dostęp do określonego przez dostawcę zbioru usług \cite{azure-students}.
Stąd platforma Azure została wybrana do prototypowania.

\subsubsection{Zbieranie danych w chmurze --- Azure IoT Hub}
Jako \texttt{Platformy IoT} użyto usługi IoT Hub. Pozwala ona na rejestrowanie
wielu urządzeń IoT, zarządzanie ich uprawnieniami, zbieranie danych i wysyłanie
wiadomości do urządzeń. W portalu zarejestrowano obie wykorzystane bramy IIoT, co przedstawia
zrzut ekranu zamieszczony na rys. \ref{fig:iot_hub_1}. W ustawieniach
usługi IoT Hub zdefiniowano także trasowanie (ang. \emph{message routing})
na wejście usług Blob Storage i Functions pełniących rolę komponentu
\texttt{Bazy Danych} i \texttt{Kompunentu Przetwarzającego}.

\begin{figure}[h]
      \centering
      \fbox{\includegraphics[width=0.8\textwidth]{iot_hub_1.png}}
      \caption{Zrzut ekranu przedstawiający tabelę urządzeń zarejestrowanych w usłudze IoT Hub}
      \label{fig:iot_hub_1}
\end{figure}

\subsubsection{Składowanie w chmurze i udostępnianie --- Azure Blob Storage}
Baza Blob Storage umożliwia elastyczne przechowywanie danych zebranych w
usłudze IoT Hub w formie obiektów JSON zapisanych w strukturze plików i folderów,
którą można swobodnie przeglądać w portalu Azure (interfejs graficzny przypomina
menadżera plików).
W prototypowym systemie baza Blob Storage przechwuje zagregowane wartości
parametrów temperatury \texttt{AirMeasurement}, aktualne wskazania przepływomierzy
i czujnika poziomu cieczy \texttt{FluidControlMeasurement}, a także liczbę
produktów wytworzonych bez wad i liczbę wszystkich produktów na potrzeby
wyznaczania wartości parametru FTQ. Pobieranie i zapis danych przez systemy
zewnętrzne umożliwia interfejs programistyczny
REST z wykorzystaniem protokołu HTTP \cite{blob-storage-doc}
W ten sposób realizowane są wymagania składowania i udostępniana danych pomiarowych.

\subsubsection{Przetwarzanie w chmurze --- Azure Functions}

Dane zbierane za pośrednictwem usługi IoT Hub są także przekazywane na
wejście usługi Azure Functions realizującej wymaganie przetwarzania danych.
Umożliwia ona impelemtację prostych funkcji bezserwerowych w wybranych językach programowania.
Programista zajmuje się jedynie implementacją logiki, zaś dostarczenie niezbędnych
zasobów informatycznych i ich skalowanie zapewnia chmura.
W ramach prototypu napisano funkcję w języku Javascript, która wyznacza
wartość parametru FTQ. Na wejściu przyjmuje ona wiadomość informującą o wyniku
oceny jakości nowo wytworzonego produktu w postaci \texttt{ProductEvaluationMeasurement}.
Z bazy Blob Storage pobierane są ostatnie wartości liczników produktów wyprodukowanych
bez wad oraz wszystkich produktów. Odpowiedni licznik jest inkrementowany
i wyznaczana jest nowa wartość parametru FTQ na podstawie wzoru \ref{eq:ftq}
zamieszczonego w rozdziale~\ref{scenariusze}. Następnie aktualizowane są wpisy
w bazie i wynik jest przesyłany do chmury InfluxDB na potrzeby bieżącej wizualizacji.
Kod źródłowy funkcji przedstawiono na listingu~\ref{lst:azure-fun}. Zdefiniowano
także drugą, prostszą funkcję pomocniczą, która przekazuje wartości parametru \texttt{FluidControlMeasurement}
do bazy InfluxDB. Zbierająca wartości tego parametru brama IOT2020
nie wspiera bowiem na chwilę obecną oprogramowania
umożliwiającego bezpośredni przesył danych z Node-RED do chmury InfluxDB, stąd
konieczne było wykorzystanie pośrednika w psotaci Azure IoT Hub i Functions.

\begin{lstlisting}[caption={Kod źródłowy funkcji wyznaczającej aktualną wartość 
      parametru FTQ osadzonej w usłudze Azure Functions}, label={lst:azure-fun}]
IoTHubMessages.forEach(message => {
      message = JSON.parse(message);
 
      // akceptacja wiadomosci o okreslonym typie
      if (message.type !== 'product-evaluation-result') {
          return;
      }
  
      // wczytanie danych z Blob Storage
      input = context.bindings.inputBlob;
      var ftqParams;

      // sprawdzenie, ktorego produktu dotyczy wiadomosc
      switch(message.productName) {
          case 'PRODUCT-A':
              ftqParams = input.prodA;
              break;
          case 'PRODUCT-B':
              ftqParams = input.prodB;
              break;
          case 'PRODUCT-C':
              ftqParams = input.prodC;
              break;
      }

      // inkrementacja licznikow
      incrementFtqCounters(ftqParams, message.accepted);

      // wyznaczenie wartosci ftq
      const ftq = ftqParams.positive / ftqParams.all * 100;

      // przeslanie wyniku do bazy InfluxDB
      sendToInflux(message.productName, ftq, context);

      // zapis wyniku do Blob Storage
      context.bindings.outputBlob = input;
});   
\end{lstlisting}

\subsubsection{Wizualizacja danych --- InfluxDB i Grafana}

Do realizacji wymagania wizualizacji danych użyto bazy danych InfluxDB.
Jest to baza zoptymalizowana pod kątem przechowywania szeregów czasowych
(ang. \emph{Time Series Database}). InfluxDB to bezpłatne
oprogramowanie dostępne na licencji MIT, które można zainstalować na dowolnej
maszynie z systemem Linux, Mac OS X, Windows. Dostępna jest także
płatna usługa chmurowa o nazwie InfluxDB Cloud firmy InfluxData. Usługa ta oferuje hosting
bazy w chmurze wraz z panelem zarządzania dostępnym z przeglądarki jako
aplikacja internetowa. Osadzenie bazy w chmurze umożliwia łatwą konfigurację
i skalowanie ,,na żądanie`` bez konieczności zajmowania się szczegółami technicznymi.
Dostawca przewodział darmowy program startowy, co wykorzystano w niniejszym projekcie
na potrzeby prototypowania. Dane pomiarowe są składowane w InfluxDB
jako rekordy w tabelach pogrupowanych w konternery --- tzw. wiadra (ang. \emph{buckets}).
Do podstawowych kolumn należą: znacznik czasowy pomiaru (\texttt{\_time}),
wartość parametru (\texttt{\_value}), nazwa parametru (\texttt{\_field}) i nazwa pomiaru (\texttt{\_measurement})
\cite{influx-db-cloud}, \cite{influx-db}.
Zrzuty ekranów z tabel InfluxDB przechowujących pomiary wilgotności i temperatury
przedsawiono na rysunkach \ref{fig:influx_1} i \ref{fig:influx_2}.

\begin{figure}[h]
      \centering
      \fbox{\includegraphics[width=0.9\textwidth]{influxDB_1.png}}
      \caption{Zrzut ekranu przedstawiający tabelę InfluxDB zawierającą szeregi czasowe pomiaru temperatury}
      \label{fig:influx_1}
\end{figure}

\begin{figure}[h]
      \centering
      \fbox{\includegraphics[width=0.9\textwidth]{influxDB_2.png}}
      \caption{Zrzut ekranu przedstawiający tabelę InfluxDB zawierającą szeregi czasowe pomiaru wilgotności}
      \label{fig:influx_2}
\end{figure}

Z bazą InfluxDB dobrze integruje się Grafana --- darmowa aplikacja do wizualizacji
danych dostępna na licencji Apache 2.0. Podobnie jak
w przypadku InfluxDB, możliwa jest lokalna instalacja lub skorzystanie z
płatnej usługi chmurowej. W ramach prototypu narzędzie zostało zainstalowane
na komputerze klienckim. Jest to serwer, który udostępnia interfejs graficzny
w postaci aplikacji przeglądarkowej, w ramach której można tworzyć
dowolne diagramy, wykresy, tabele, wskaźniki i swobodnie rozbudowywać
modularne panele zarządzania. Istnieje także możliwość tworzenia własnych komponentów
graficznych i wtyczek.
Grafana komunikuje się z bazą InfluxDB za pośrednictwem Internetu z wykorzystaniem
deklaratywnego języka zapytań Flux. Przykładowe zapytanie przedstawiono na
listingu \ref{lst:flux}. Wykonywana jest selekcja
rekordów z kontenera o nazwie \texttt{bucket}, w których wartość pola
\texttt{\_measurement} to \texttt{freezer}, a pola \texttt{\_field} --- \texttt{temperature}.
Zakres wyników ograniczany jest do trzech ostatnich godzin, a wartości
są agregowane z wykorzystaniem funkcji wyznaczającej średnią (\texttt{mean}) w obrębie interwałów
pięciosekundowych. Wynik zapytania może być wizualizowany przez wybrany komponent
graficzny. Na rys. \ref{fig:grafana-1} zamieszczono zrzut ekranu będący fragmentem panelu Grafana,
przedstawiający wykres i wskaźnik wartości wizualizujące wynki zapytania z listingu \ref{lst:flux}.


\begin{lstlisting}[caption={Przykładowe zapytanie w języku Flux}, label={lst:flux}]
from(bucket:"bucket")
      |> range(start:-3h)
      |> filter(fn:(r) =>
            r._measurement == "freezer" and
            r._field == "temperature"
      )
      |> aggregateWindow(every: 5s, fn: mean)
\end{lstlisting}

\begin{figure}[h]
      \centering
      \fbox{\includegraphics[width=0.9\textwidth]{grafana_1.png}}
      \caption{Zrzut ekranu z panelu Grafana wizualizujący wyniki zapytania przedstawione na listing \ref{lst:flux}}
      \label{fig:grafana-1}
\end{figure}

\clearpage

\section{Specyfikacja zewnętrzna}\label{spec-zew}

\subsection{Wdrożenie systemu}

\subsubsection{Uruchomienie i konfiguracja bram IIoT}

W celu uruchomienia urządzeń pełniących rolę bram IIoT należy w pierwszej
kolejności przygotować środowisko ich funkcjonowania, tj. wydzielić
miejsce ich instalacji zapewniając łatwy dostęp i odpowiedni poziom zabezpieczeń
przed czynnikami niesprzyjającymi lub osobami nieuprawnionymi. Następnie
konieczne jest dostarczenie wymaganego źródła zasilania określonego
w specyfikacji urządzenia. Kolejny etap polega na konfiguracji oprogramowania
(w tym ewentualnego systemu operacyjnego) i połączeń sieciowych.

W ramach prototypu użyto dwóch bramy IIoT - Simatic IOT2020 i Raspberry Pi 3 B.
Brama IOT2020 wymaga stałego napięcia zasilania mieszczącego się w zakresie 9 - 36V.
Pobór prądu wynosi 1.4A. Dla RaspberryPi producent zaleca użycie
zasilacza sieciowego microUSB o napięciu wyjściowym 5.1V i prądzie 2.5A.
Przy wyborze źródła zasilania należy zwrócić uwagę na rodzaj obudowy. Warto
rozważyć produkty posiadające odpowiedni stopień ochrony (np. IP20) i możliwość
montażu na szynie DIN (co jest rozwiązaniem popularnym w przemyśle).

Dla każdej z bram producenci przygotowali dedykowaną dystrybucję systemu Linux.
Obrazy systemów należy najlepiej pobrać z oficjalnych stron i zainstalować na kartach
microSD o minimalnej pojemności 8 GB. Po umieszczeniu karty w odpowiednim
slocie urządzenia i podłączenia zasilania system uruchamia się samoczynnie.
Następnie należy dokonać podstawowej konfiguracji systemu operacyjnego: podłączenia i konfiguracji interfejsów sieciowych,
wprowadzenia użytkowników i nadania praw, instalacja preferowanych narzędzi.

Podstawą oprogramowania bram w przyjętej implementacji prototypu jest
narzędzie Node-RED wymagające zainstalowanego środowiska Node.js. W przypadku
obu obrazów systemów oprogramowanie to jest dostępne domyślnie. Dla RaspberryPi
warto rozważyć natychmiastową aktualizację. W przypadku IOT2020 aktualizacja
może być utrudniona, ze względu na ograniczone wsparcie Node.js dla procesora
Intel Quark x1000. W tym przypadku najlepiej jest śledzić aktualności na
forum producenta.

Serwer Node-RED należy uruchomić zgodnie z dokumentacją. Domyślny port, na
którym nasłuchuje serwer to 1880. Po wpisaniu w przeglądarce lokalnego adres i portu
w oknie pojawia się widok aplikacji do zarządzania serwerem Node-RED. Aplikacja
umożliwia tworzenie i wdrażanie programów, dodawanie bibliotek, importowanie i eksportowanie projektu.
Instalację bibliotek (w szczególności własnych, lokalnych węzłów) umożliwia także menadżer pakietów \texttt{npm}.
W zależności od integrowanych interfejsów i protokołów konieczne jest ustawienie
określonych parametrów połączeń, np. adres serwera pośrednika (brokera) w przypadku protokołu MQTT.

Dalsze czynności polegają na rozwoju i utrzymaniu bram w zależności od wymagań
biznesowych. Działania te sprowadzają się m. in. do dołączania aparatury pomiarowej dostarczającej dane
z wykorzystaniem dostępnych interfejsów, implementacji logiki przetwarzania
brzegowego w ramach Node-RED, instalacji dodatkowego oprogramowania
(w przypadku wykonanego prototypu zainstalowano przykładowo serwer pośrednika MQTT Mosquitto na RaspberryPi).


\subsubsection{Konfiguracja usług w chmurze}

Każda chmura posiada nieco inny sposób konfiguracji i model zarządzania.
Czynności administracyjne, które należy przeprowadzić, żeby korzystać z usług
chmurowych, to zwykle: rejestracja konta organizacji i użytkowników,
wybór modelu rozliczeniowego (subskrypcji), konfiguracja i wzajemna
integracja zasobów, które można postrzegać jako instancje usług. Przykładowe zasoby
to bazy danych, funkcje bezserwerowe, maszyny wirtualne, itp.

Na potrzeby prototypu postanowiono użyć chmury Azure wraz z usługami:
IoT Hub, Blob Storage, Functions. Po założeniu konta i uzyskaniu
dostępu do portalu należy stworzyć grupę zasobów i dołączyć do niej nowy zasób
w ramach każdej z wymienionych usług.
Konfiguracja zasobu IoT Hub polega m. in. na zarejestrowaniu urządzeń (w tym przypadku bram IIoT).
W celu skorzystania z usługi Blob Storage należy stworzyć zasób typu
Storage Account i zdefiniować kontenery na dane zgodnie z zapotrzebowaniem.
Przechodząc do wybranego kontenera można przeglądać składowane dane, które
są zorganizowane w strukturę folderów i plików.
Użycie usługi Azure Functions wymaga stworzenia zasobu typu Function App.
Nowe funkcje można implementować przez panel zasobu Function App za pośrednictwem
edytora w przeglądarce (tylko dla języka JavaScript). Dla większej wygody
kod źródłowy można rozwijać lokalnie w jednym ze wspieranych języków programowania
i przesyłać do chmury z wykorzystaniem narzędzia Function Core Tools.
Żeby umożliwić wyzwalanie funkcji przez wiadomości (dane) zbierane w usłudze IoT Hub,
a także zapis danych do bazy Blob Storage należy w ustawieniach IoT Hub
skonfigurować trasowanie (integrację zasobów) przez zakładkę o nazwie Message Routing.

W implementacji projektu postanowiono także wykorzystać bazę InfluxDB
osadzoną w chmurze InfluxDB Cloud. Po założeniu konta w usłudze i wybraniu
modelu rozliczeniowego można od razu przejść do zarządzania zainstalowaną
i uruchomioną już bazą za pośrednictwem aplikacji przeglądarkowej. Panel
umożliwia m. in. definiowanie kontenerów na dane, edycję ustawień,
realizację zapytań do bazy wraz z podstawową wizualizacją wyników.


\subsubsection{Instalacja i konfiguracja narzędzia do wizualizacji}\label{grafana-conf}

Na rynku dostępnych jest wiele darmowych i płatnych narzędzi do wizualizacji danych.
Sposób ich instalacji i konfiguracji jest zróżnicowany. Mogą to być mianowicie
aplikacje przeznaczone do instalacji lokalnej na komputerze osobistym,
statyczne albo dynamiczne aplikacje internetowe (ang. \emph{Static Web Applications} / \emph{Single Page Applications})
z oprogramowaniem serwera (ang. \emph{backend}), a także
aplikacje na platformy mobilne i urządzenia wbudowane.

Do realizacji prototypu wybrano aplikację Grafana, jako że jest to darmowe,
modularne (łatwe w rozbudowie) narzędzie, które domyślnie wspiera
bazę InfluxDB. Aplikację postanowiono zainstalować lokalnie na komputerze osobistym
z systemem Linux pobrawszy pakiet instalacyjny ze strony producenta.
Alternatywnie Grafana może być zainstalowana na maszynie wirtualnej w
chmurze albo można skorzystać bezpośrednio z płatnej usługi chmurowej producenta
o nazwie Grafana Cloud (analogicznie jak w przypadku InfluxDB Cloud).
Sposób użytkowania aplikacji jest podobny do Node-RED. Działający w tle
serwer udostępnia aplikację przeglądarkową na domyślnym porcie 3000, w ramach której
można konfigurować użytkowników, dodawać źródła danych i tworzyć panele wizualizacji.
Żeby móc odczytywać dane z bazy InfluxDB należy w ustawieniach dodać źródło danych
typu InfluxDB, podać adres URL bazy, nazwę organizacji i kontenera, a także wpisać
znacznik (ang. \emph{token}), który można wygenerować w panelu
zarządzania bazą InfluxDB. Znacznik to specjalny, poufny ciąg znaków, który
umożliwia klientom (aplikacjom i/lub użytkownikom) uwierzytelnianie i autoryzację.


\subsubsection{Integracja bram IIoT z usługami chmurowymi}

W celu połączenia urządzeń z usługami chmurowymi należy w pierwszej kolejności
dokonać ich rejestracji w panelu zarządzania chmurą, a następnie zaimplementować
wymagany protokół po stronie urządzenia lub skorzystać z dedykowanej biblioteki
będącej wygodną w użyciu fasadą dla protokołu. Bibliotekę należy odpowiednio
skonfigurować, co odbywa się zwykle przez określenie parametrów połączenia:
adresu IP albo URL, portu, danych uwierzytelniających, itd.

Użyte na potrzeby prototypu oprogramowanie bram IIoT w postaci Node-RED
posiada w swoim katalogu darmowe biblioteki umożliwiające połączenie zarówno z usługą
Azure IoT Hub (\texttt{node-red-contrib-azure-iot-hub }),
jak i bazą InfluxDB osadzoną w chmurze (\texttt{node-red-contrib-influxdb}).
Węzeł IoT Hub wymaga dostarczenia danych połączenia w postaci nazwy hosta,
identyfikatora urządzenia i klucza dostępu, które można znaleźć w ustawieniach
zarejestrowanego urządzenia w portalu Azure. Węzeł służący do połączenia z
InfluxDB Cloud wymaga dostarczenia tych samych danych, co w przypadku integracji
aplikacji Grafana (patrz: rozdział \ref{grafana-conf}).

\subsection{Kategorie użytkowników}

Na diagramie przypadków użycia (patrz: rozdział \ref{use-case}, rys. \ref{fig:use-case})
przedstawiono użytkowników systemu wraz z możliwym zakresem ich działań.
W systemie wyróżniono następujące kategorie użytkowników:
\begin{itemize}
      \itemsep0em
      \item Administratorzy --- Są to pracownicy zarządzający infrastrukturą systemu.
            Do ich zadań należy instalacja, konfiguracja i utrzymanie urządzeń,
            zarządzanie sieciami komputerowymi, dbanie o bezpieczeństwo (monitorowanie,
            wykrywanie awarii, aktualizacja oprogramowania i certyfikatów),  zarządzanie zasobami w chmurze
            i monitorowanie kosztów, zarządzanie użytkownikami, administracja urządzeniami klienckimi.
            W zależności od skali systemu administratorzy mogą pełnić obowiązki
            w obrębie całego systemu lub poszczególnych jego warstw i~komponentów;
      \item Programiści i testerzy --- Implementują i testują logikę przetwarzania
            danych osadzaną na urządzeniach warstwy brzegowej oraz w chmurze;
      \item Analitycy, pracownicy obsługujący proces przemysłowy i~inni ---
            Kategoria odnosi się do wszystkich użytkowników, którzy korzystają z wników zbierania, przetwarzania
            i wizualizacji danych, a także uczestniczą w rozwoju logiki przetwarzania danych.
\end{itemize}

\subsection{Kwestie bezpieczeństwa}

\subsubsection{Niezawodność}

Zagadnienie niezawodności systemu posiada dwa aspekty --- sprzętowy i programowy.
Żeby zapewnić poprawne działanie w warstwie sprzętowej, należy w pierwszej
kolejności dokładnie przemyśleć dobór urządzeń. Ich specyfikację należy zestawić
z potencjalnymi warunkami, w których przyjdzie ich pracować. Te zaś w środowisku
przemysłowym mogą być różne. Mogą w nim bowiem występować zakłócenia elektromagnetyczne,
mechaniczne i chemiczne mające realny wpływ na pracę urządzeń elektronicznych.
Ponadto dla uzyskania większego poziomu bezawaryjności można rozważyć
instalację źródeł awaryjnego zasilania, wprowadzenie urządzeń nadmiarowych (redundantnych)
przejmujących zadania uszkodzonych urządzeń, a także układów typu \emph{watchdog}
wykrywających awarie.

Drogą do osiągnięcia niezawodności w warstwie programowej jest przede wszystkim sporządzanie
odpowiedniej liczby testów automatycznych oraz przeprowadzenie testów manualnych
na poziomie jednostkowym, integracyjnym i systemowym.
Można również rozważyć użycie programów i/lub urządzeń typu \emph{watchdog}
wykrywających potencjalne awarie i wyzwalające przewidziane reakcje systemu,
na przykład ponowne uruchomienie urządzeń, próba ponownego nawiązania połączenia, itp.

\subsubsection{Ograniczenie dostępu}

Kwestie zapewnienia dostępu do systemu wyłącznie osobom i aplikacjom do tego
uprawnionym można również rozpatrywać na płaszczyźnie sprzętowej i programowej.
Dla warstwy sprzętowej należy rozważyć wydzielenie specjalnych pomieszczeń
dla używanych urządzeń, do których dostęp będzie kontrolowany i monitorowany
przez wyznaczony do tego personel i odpowiednie systemy monitoringu.

Ograniczenie dostępu do płaszczyzny programowej polega przede wszystkim na zastosowaniu
metod uwierzytelniania i autoryzacji oraz wykorzystaniu sieciowych mechanizmów zabezpieczeń
(zapora ogniowa, oprogramowanie antywirusowe, itd.). Uwierzytelnianie i autoryzacja z wykorzystaniem
identyfikatorów i haseł powinny znaleźć zastosowanie zarówno w warstwie brzegowej,
jak i biznesowej systemu. Wymagane jest, aby hasła posiadały odpowiedni stopień skomplikowania
i były często aktualizowane.

Uwierzytelnianiu i autoryzacji w obrębie zrealizowanego prototypu podlega dostęp do: systemów operacyjnych
bram IIoT, aplikacji Node-RED, portali zarządzania chmurą Azure i InfluxDB Cloud,
komputerów klienckich i aplikacji Grafana. Komponenty integrujące się z chmurą
(Node-RED i Grafana) uzyskują dostęp na podstawie wspomnianego już znacznika ---
specjalnego ciągu znaków pełniącego rolę klucza. Znaczniki można w każdej
chwili dezaktywować lub wycofywać poszczególne uprawnienia. Analogicznie przedstawia się
sytuacja z kontami użytkowników w chmurach. Nieco więcej wysiłku może wymagać
administracja kontami na poziomie urządzeń przedsiębiorstwa. W niektórych
przypadkach możliwe jest wykorzystanie mechanizmu pojedynczego logowania
(ang. \emph{Single Sign-On}, w skrócie: SSO) zintegrowanego z chmurą, np.
w ramach usług Azure Active Directory czy AWS Directory Service.
Dla podniesienia poziomu bezpieczeństwa można także rozważyć wprowadzenie
mechanizmów ochrony dostępu na poziomie protokołów używanych do komunikacji
w warstwie brzegowej, np. ustawienie hasła w serwerze pośredniczącym MQTT.

\subsubsection{Szyfrowanie komunikacji}

W celu zwiększenia ochrony poufności danych przesyłanych za pośrednictwem sieci
komputerowych należy wdrożyć odpowiednie mechanizmy szyfrowania. Jest to aspekt
szczególnie ważny, gdy ma miejsce wymiana danych w ramach sieci publicznej --- Internetu.
W kontekście realizowanego projektu dotyczy to połączenia bram IIoT z usługami chmurowymi.
Wśród dostępnych rozwiązań wyróżnia się m. in. użycie protokołu kryptograficznego TLS w połączeniu
ze standardowymi protokołami HTTP, MQTT, itd. Przydatne może okazać się również
wykorzystanie mechanizmu tunelowania w postaci usługi VPN (ang. \emph{Virtual Private Network}).
Powinno się także rozważyć szyfrowanie komunikacji na poziomie sieci warstwy brzegowej.
W tym przypadku można jednak napotkać pewne problemy, gdyż nie wszystkie urządzenia
są w stanie implementować algorytmy kryptograficzne ze względu
na ograniczone zasoby, jak również nie wszystkie protokoły posiadają specyfikację
odnoszącą się do zagadnień bezpieczeństwa.

Wykorzystane na potrzeby prototypu chmury Azure i InfluxDB narzucają szyfrowanie
komunikacji z wykorzystaniem protokołu TLS. Narzędzia Node-RED i Grafana
umożliwiają konfigurację certyfikatów SSL i tym samym użycie protokołu
HTTPS do komunikacji z serwerem aplikacji w ramach przeglądarki. Specyfikacje
protokołów MQTT, CoAP i Modbus/TCP również przewidują użycie protokołu TLS albo
DTLS (w przypadku, gdy protokołem warstwy transportowej jest UDP).

\newpage
\section{Weryfikacja i walidacja}\label{testy}

W niniejszym rozdziale opisano dwa rodzaje testów przeprowadzone 
w celu weryfikacji i walidacji działania systemu: testy jednostkowe oraz systemowe.

\subsection{Testy jednostkowe}

W celu weryfikacji poprawności działania autorskich komponentów, tj.
funkcji bezserwerowych uruchamianych w chmurze Azure oraz Bufora Brzegowego i funkcji przetwarzających Node-RED
zaimplementowano testy jednostkowe. Żeby uruchomić i zautomatyzować przeprowadzanie
testów posłużono się darmowymi narzędziami: dedykowanym modułem \texttt{node-red-node-test-helper}
w przypadku Node-RED oraz narzędziem Jest w przypadku funkcji Azure implementowanych
w języku JavaScript.

\subsection{Testy systemowe}\label{testy-systemowe}

Na potrzeby weryfikacji poprawności działania całego systemu oraz sprawdzenia,
czy spełnia on zdefiniowane wymagania (walidacja), uruchomiono w warunkach
domowych zrealizowany prototyp systemu o konfiguracji przedstawionej na rys. \ref{fig:network}.
Zbudowano sieć komputerową na bazie Ethernet i WiFi składającą się z dwóch
podsieci: 10.42.0.0/24 przeznaczona dla urządzeń przewodowych oraz
192.168.1.0/24, w ramach której komunikują się urządzenia bezprzewodowe.
Bramy RaspberryPi oraz Simatic IOT2020 nie pełnią w tej konstelacji roli
bram w sensie sieci, lecz w sensie funkcjonalnym bram IIoT, czyli gromadzenia danych, przetwarzania
brzegowego, translacji protokołów i przekazywania danych do chmury.
Właściwą bramą sieciową udostępniającą połączenie z Internetem jest w systemie urządzenie o adresie 10.42.0.1.

\begin{figure}[h]
      \centering
      \includegraphics[width=0.9\textwidth]{network.png}
      \caption{Schemat ideowy systemu uruchomionego na potrzeby testów systemowych}
      \label{fig:network}
\end{figure}

Na komputerach o adresach 192.168.1.3, 10.42.0.4 i 10.42.0.5 zainstalowano i uruchomiono
autorskie oprogramowanie symulujące urządzenia pomiarowe, które generuje kolejne dane
i przesyła je do bram z wykorzystaniem odpowiednich protokołów. Program napisano
w języku Java. Zaprojektowano prostą architekturę przedstawioną na rys. \ref{fig:iot-sim}.
Rozszerzanie działania symulatora odbywa się przez implementację interfejsów
i klasy abstrakcyjnej znajdujących się na diagramie. Implementacje interfejsu \texttt{MeasurementSimulator}
mają za zadanie generować kolejne wartości pomiarów w postaci obiektów typu
\texttt{Measurement}. Otrzymane dane są wysyłane do odbiorcy z wykorzystaniem implementacji
interfejsu \texttt{MeasurementSender}. Do implementacji protokołów wykorzystano
darmowe biblioteki: Eclipse Paho (MQTT), Apache PLC4J (Modbus/TCP), Eclipse Californium (CoAP).

\begin{figure}[h]
      \centering
      \includegraphics[width=0.8\textwidth]{iot-sim-architecture.png}
      \caption{Schemat ideowy systemu uruchomionego na potrzeby testów systemowych}
      \label{fig:iot-sim}
\end{figure}

Implementacja symulatora parametrów powietrza dla każdego z trzech pomieszczeń 
pozwala na zdefiniowanie listy kroków
symulacji, które wykonywane przez określoną liczbę wywołań metody
\texttt{generateNext}. Dla każdego kroku należy zdefiniować jeden z trybów pracy:
\begin{itemize}
      \itemsep0em
      \item \texttt{OSCILLATE} --- pseudolosowa oscylacja (w ograniczonym przedziale),
      \item \texttt{DROP} --- spadek o pseudolosową wartość (z~ograniczonego przedziału),
      \item \texttt{GROW} --- wzrost o pseudolosową wartość (z~ograniczonego przedziału).
\end{itemize}
\noindent W ten sposób możliwe jest definiowanie przebiegów o konkretnym kształcie, aczkolwiek
wyglądającym naturalnie ze względu na pewien stopień losowości. Wygenerowane 
dane są przesyłane cyklicznie do serwera pośredniczącego MQTT znajdującego
się pod adresem 192.168.1.2:1883 (RaspberryPi).

Symulator danych dotyczących jakości wytworzonych produktów przyjmuje jako parametr
oczekiwaną wartość parametru FTQ, która określa prawdopodobieństwo wystąpienia
produktu spełniającego wymagania jakościowe. Wartość wskaźnika zbliża się więc
w każdym kroku symulacji do zadanej. Wygenerowane dane są przesyłane cyklicznie
do serwera CoAP znajdującego się pod adresem 192.168.1.2:5683 (RaspberryPi.)

Symulator stanu przepływomierzy i poziomu cieczy w zbiorniku głównym generuje 
w każdym kroku symulacji nowy stan dla przepływomierzy. Od poziomu cieczy w zbiorniku głównym
odejmowana jest następnie wartość ostatniego przyrostu zużycia cieczy w każdej ze stref. 
Zbiornik jest automatycznie uzupełniony po przekroczeniu zdefiniowanej wartości minimalnej.
Symulator pełni zatem jednocześnie funkcję atrapy sterownika PLC. Cztery wygenerowane
wartości są zapisywane cyklicznie do węzła podrzędnego Modbus/TCP nasłuchującego
pod adresem 10.42.0.5:502 (na tym samym komputerze, co symulator). Dane 
z atrapy PLC są odczytywane przez bramę Simatic IOT2020 będącą węzłem nadrzędnym, 
znajdującą się pod adresem 10.42.0.3. 

Testy systemu polegały na obserwacji wizualizacji w narzędziu Grafana i weryfikacji,
czy otrzymane wyniki są zgodne z wartościami wygenerowanymi przez symulator.
W pierwszej kolejności skonfigurowano symulatory do wygenerowania wartości stałych,
zapisanych ,,na sztywno``, żeby jednoznacznie stwierdzić poprawność. Następnie 
uruchomiono opisane wcześniej przebiegi symulacji na czas trzech godzin. 
Oczekiwano następujących rezultatów:
\begin{itemize}
      \itemsep0em
      \item określony kształt przebiegu wartości temperatury i wilgoci dla każdej 
      ze stref (odpowiednio: prostokątny, trapezoidalny i trójkątny),
      \item zbliżanie się wartości parametrów FTQ do zadanych,
      \item stopniowy wzrost wskazań stanu przepływomierzy i adekwatny poziom cieczy 
      w~zbiorniku głównym. 
\end{itemize}
\noindent Obok obserwacji wizualizacji dokonano także przeglądu danych zapisywanych
w~bazach Azure Blob Storage i InfluxDB. 

Opisane testy przeprowadzano wielokrotnie, a z każdą iteracją wykrywano 
pewne błędy wynikające z:
\begin{itemize}
      \itemsep0em
      \item błędnej konfiguracji połączeń sieciowych (nieprawidłowy adres IP, maska, brama sieciowa),
      \item błędów w implementacji logiki przetwarzania w warstwie brzegowej,
      \item błędnej konfiguracji węzłów odpowiedzialnych za integrację protokołów i komunikację z chmurą,
      \item niepoprawnej konfiguracji trasowania wiadomości z Azure IoT Hub do innych usług,
      \item błędów w logice funkcji bezserwerowych, 
      \item błędów w składni języka zapytań Flux,
      \item niepoprawnej konfiguracji komponentów w aplikacji Grafana.  
\end{itemize}
\noindent W ostateczności wszystkie wymienione usterki zostały poprawione i więcej nieprawidłowości nie
stwierdzono. 

Przeprowadzone testy pozwoliły na weryfikację działania systemu, a także na
walidację --- sprawdzenie, czy spełnione są zdefiniowane wymagania. Prototypowa 
konfiguracja realizuje bowiem zadania zbierania, przetwarzania, składowania,
udostępniania i wizualizacji danych pomiarowych. Wykorzystano określone rozwiązania
IoT i~usługi chmurowe. Użycie bramy Simatic IOT2020 i protokołu Modbus pozwoliło na dostosowanie 
do środowiska przemysłowego w zakresie podstawowym. Oprogramowanie Node-RED
realizuje wymaganie ogólności, jako że umożliwia integrację różnych urządzeń i protokołów.
Usługi chmurowe pozwalają na stosunkowo proste skalowanie zasobów. Zapewniono
podstawowy poziom bezpieczeństwa --- dostęp do komponentów systemu wymaga uwierzytelnienia
z użyciem hasła, zaś komunikacja za pośrednictwem Internetu odbywa się z wykorzystaniem mechanizmów
szyfrowania.

\section{Podsumowanie}\label{wnioski}

W ramach projektu podjęto zadanie wypracowania ogólnej koncepcji systemu
wraz z przykładową implementacją w postaci prototypu, którego przeznaczeniem jest
zbieranie danych pomiarowych w środowisku przemysłowym, a także ich składowanie, przetwarzanie,
udostępnianie i wizualizacja. Przyjęto założenie, że należy w tym celu wykorzystać możliwości
oferowane przez IoT i chmurę obliczeniową. W rozdziale \ref{ogolna-koncepcja}
przedstawiono uniwersalny model architektury systemu, który opracowano na podstawie
dostępnego w literaturze trójwarstwowego wzorca systemów IIoT oraz referencyjnego
modelu architektury RAMI 4.0 (patrz: rozdział \ref{iiot}). Zaproponowany model
posiada strukturę komponentową podzieloną na trzy grupy ze względu na miejsce wdrożenia:
środowisko przemysłowe (bramy IIoT), chmura i urządzenia klienckie. 
Każdy z komponentów jest odpowiedzialny za określony
zakres funkcjonalny, może być swobodnie wymieniany, modyfikowany i posiadać różne implementacje.
Dla poszczególnych elementów architektury przedstawiono potencjalne realizacje
na podstawie przeglądu dostępnych urządzeń, rozwiązań technicznych i  narzędzi informatycznych. 

Żeby umożliwić weryfikację działania systemu, 
na potrzeby prototypu zdefiniowano konkretne scenariusze biznesowe, które 
podlegały implementacji: pomiar parametrów powietrza (temperatury i wilgotności),
wyznaczanie wskaźnika jakości produkcji na podstawie ewaluacji produktów, 
wizualizacja stanu zużycia chemii przemysłowej. Wykorzystano urządzenia
RaspberryPi i Siemens Simatic IOT2020, które pełnią rolę bram IIoT. Zbieranie
danych pomiarowych w systemie odbywa się z wykorzystaniem sieci Ethernet oraz WiFi.
Do komunikacji użyto charakterystycznych dla IoT protokołów MQTT i CoAP, a także
popularnego w przemyśle protokołu Modbus. Integrację różnych interfejsów i protokołów
umożliwia darmowe oprogramowanie Node-RED zainstalowane na bramach IIoT. 
Realizuje ono przetwarzanie brzegowe i przesył danych do chmury za pośrednictwem Internetu.
Pozwala na elastyczną modyfikację i rozbudowę logiki przetwarzania.
W warstwie biznesowej systemu użyto chmury Microsoft Azure. W celu 
gromadzenia danych przesyłanych z bram użyto usługi IoT Hub, zaś dalsze
przetwarzanie, składowanie i udostępnianie danych realizują usługi Functions i Blob Storage
oraz baza danych szeregów czasowych InfluxDB osadzona w chmurze InfluxDB Cloud. 
Do wizualizacji danych użyto darmowego narzędzia Grafana, które integruje się z bazą InfluxDB i
umożliwia tworzenie wykresów, diagramów, tabel, wskaźników, jak również własnych
komponentów graficznych.

Wśród napotkanych wyzwań projektowych należy przede wszystkim wymienić konieczność
połączenia wiedzy ze stosunkowo szerokiej dziedziny, którą stanowią: urządzenia wbudowane, 
systemy pomiarowe, sieci komputerowe, przemysłowe systemy informatyczne, aplikacje internetowe, IoT i chmura. 
Zasadniczym problemem technicznym okazała się być integracja komponentów systemu, co 
można wnioskować na podstawie wychwyconych błędów wymienionych w rozdziale 
\ref{testy-systemowe}. Dobrym podejściem okazało się prototypownie z wykorzystaniem
mikrokomputera RaspberryPi w pierwszej kolejności. Choć nie jest to urządzenie 
dedykowane dla przemysłu, to jednak stosunkowo duża moc obliczeniowa i 
dostępne wsparcie programistyczne przyniosły korzyści w warunkach testowych, rozwojowych.
Brama IOT2020 jest przeznaczona do zastosowań przemysłowych, lecz nie posiada 
tak dużej mocy obliczeniowej oraz wsparcia dla najnowszego oprogramowania (m. in. Node.js),
co znacząco wydłużyło czas implementacji i wdrażania, jak również wymusiło 
zrezygnowanie z niektórych bibliotek Node-RED. Zaobserwowane korzyści wynikające
z zastosowania chmury to przede wszystkim brak konieczności utrzymywania infrastruktury
informatycznej, która może być swobodnie zmieniana, rozbudowywana za pośrednictwem
portalu.  

Komponentowy model architektury systemu i wykorzystanie uniwersalnych technologii
stwarza możliwości do dalszego rozwoju. Jedną z nich jest dołączenie do prototypu
prawdziwych urządzeń aparatury przemysłowej, jak i IoT. Jeżeli wybrane urządzenia
warstwy brzegowej posiadają techniczną zdolność do pracy w czasie rzeczywistym
(np. Simatic IOT2020, BeagleBone), 
można spróbować zintegrować je z działaniem lokalnego systemu automatyki. Przed wdrożeniem 
systemu w miejscu docelowym koniecznie należy przeprowadzić testy w środowisku
możliwie najlepiej odwzorowującym środowisko produkcyjne. Warto rozpatrzyć także
zagadnienia niezawodności i bezpieczeństwa: rozbudować mechanizmy szyfrowania danych
i ograniczenia dostępu w warstwie brzegowej, wprowadzić redundancję sprzętową,
układy typu \emph{watchdog}, ewentualnie stworzyć klaster urządzeń przetwarzających 
(bram IIoT) wraz z równoważeniem obciążenia. Wykorzystując dostępne usługi 
chmurowe można zrealizować również bardziej zaawansowane przetwarzanie danych:
uczenie maszynowe i analizę \emph{big data}. Dane składowane w chmurze mogą być
ponadto udostępniane na rzecz aplikacji ,,szytych na miarę``, które pozwalają 
na realizację w zasadzie dowolnej funkcjonalności. Dostępne rozwiązania technologiczne 
umożliwiają także przepływ informacji w kierunku od chmury do warstwy brzegowej,
co może być pożyteczne w wielu scenariuszach, jednak jednocześnie podnosi 
poziom zagrożenia, jako że systemy zewnętrzne mogłyby mieć wpływ na działanie 
układów sterowania, zaś sieć publiczna ułatwia dostęp potencjalnym przestępcom. 
Implementacji takiego modelu należy zatem dokonywać bardzo rozważnie. 
Przedstawiona architektura opiera się na koncepcji bram IIoT jako pośrednika 
komunikacji. Istnieje podejście alternatywne, w myśl którego przetwarzanie brzegowe
odbywa się w chmurze, zaś urządzenia komunikują się z nią bezpośrednio. Trzeba 
jednak zwrócić uwagę na fakt, że nie wszystkie urządzenia są w stanie implementować 
kompletne stosy protokołów. Brak elementu pośredniczącego w postaci bramy utrudnia 
sprawowanie kontroli nad systemem. Stąd dla przemysłu można rozważyć 
rozwiązanie co najwyżej o charakterze hybrydowym. 

\newpage
\printbibliography

\end{document}